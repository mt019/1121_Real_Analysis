\documentclass[UTF8,a4paper,10pt]{article}

% \documentclass[UTF8,a4paper,14pt]{article}
% \usepackage[utf8]{inputenc}
\usepackage{amsmath}
% \usepackage{algorithm,algorithmic}
\usepackage[linesnumbered,ruled,vlined]{algorithm2e}

% \usepackage{algorithmicx}
% \usepackage{algpseudocode}
\usepackage{hyperref}

% \usepackage{algpseudocode}
\usepackage{amssymb}
\usepackage{amsfonts}
%for 字體
%https://tug.org/FontCatalogue/
% \usepackage[T1]{fontenc}
% \usepackage{tgbonum}
% \usepackage[bitstream-charter]{mathdesign}
% \usepackage[T1]{fontenc}
% \usepackage{bm}#粗體
%\usepackage{boondox-calo}
\usepackage{textcomp}
\usepackage{fancyhdr}%导入fancyhdrf包
\usepackage{ctex}%导入ctex包
\usepackage{enumitem} %for在Latex使用條列式清單
\usepackage{varwidth}
\usepackage{soul} %for \ul
\usepackage{comment}%\begin{comment}\end{comment}
\usepackage{cancel}%\cancel{}
%\usepackage{unicode-math}

\usepackage[dvipsnames, svgnames, x_11names]{xcolor}

\usepackage[low-sup]{subdepth}
\usepackage{subdepth}

\newcommand{\indep}{\Perp \!\!\! \Perp}

\usepackage{amsthm}
\DeclareMathOperator{\E}{\mathbb{E}}
\DeclareMathOperator{\Var}{\textbf{Var}}
\DeclareMathOperator{\Cov}{\textbf{Cov}}
\DeclareMathOperator{\Cor}{\textbf{Cor}}
\DeclareMathOperator{\X}{\mathbf{X}}
\DeclareMathOperator{\Pro}{\mathbf{P}}
\DeclareMathOperator{\M}{\mathbf{M}}
\DeclareMathOperator{\Id}{\mathbf{I}}
\DeclareMathOperator{\Y}{\mathbf{Y}}
\DeclareMathOperator{\MSFE}{\mathbf{MSFE}}
\DeclareMathOperator{\e}{\mathbb{e}}
\DeclareMathOperator{\V}{\mathbf{V}} 
\DeclareMathOperator{\tr}{\text{tr}}
\DeclareMathOperator{\A}{\textbf{A}}
\DeclareMathOperator{\diag}{diag}
\DeclareRobustCommand{\rchi}{{\mathpalette\irchi\relax}}
\newcommand{\irchi}[2]{\raisebox{\depth}{$#1\chi$}} % inner command, used by \rchi
\DeclareMathOperator*{\argmax}{arg\,max}
\DeclareMathOperator*{\argmin}{arg\,min}

\DeclareMathSizes{20}{10}{10}{5}

\usepackage[a4paper, margin=1in]{geometry}
% \setlength\parskip{5ex}% it would be better define distance in ex (5ex) 
                         %  or in pt, pc, mm, etc (see edit below)

\setlength{\parindent}{0pt}
\usepackage{array, makecell} %


%中英文設定
%\usepackage{fontspec}
% \setmainfont{TeX Gyre Termes}
% \usepackage{xeCJK} %引用中文字的指令集
% %\setCJKmainfont{PMingLiU}
% \setCJKmainfont{DFKai-SB}





% \setmainfont{Times New Roman}
% \setCJKmonofont{DFKai-SB}
\pagenumbering{arabic}%设置页码格式
\pagestyle{fancy}
\fancyhead{} % 初始化页眉
\usepackage{advdate}

% \newcommand{\yesterday}{{\AdvanceDate[-1]\today}}

\fancyhead[C]{Real Analysis\quad HW 03\quad  R10A21126\quad  WANG YIFAN\quad   \today}
%\fancyhead[LE]{\textsl{\rightmark}}
%\fancyfoot{} % 初始化页脚
%\fancyfoot[LO]{奇数页左页脚}
%\fancyfoot[LE]{偶数页左页脚}
%\fancyfoot[RO]{奇数页右页脚}
%\fancyfoot[RE]{偶数页右页脚}

% \title{{Econometrics HW 05}}
% \author{R10A21126}
% \date{\today}

%\fancyhf{}
\usepackage{lastpage}
\cfoot{Page \thepage \hspace{1pt} of\, \pageref{LastPage}}

\renewcommand{\headrulewidth}{0.1pt}%分隔线宽度4磅
%\renewcommand{\footrulewidth}{4pt}

\allowdisplaybreaks
\usepackage[english]{babel}
%\usepackage{amsthm}
\newtheorem{theorem}{Theorem}[section]
\newtheorem{corollary}{Corollary}[theorem]
\newtheorem{lemma}[theorem]{Lemma}


\usepackage[most]{tcolorbox}

\definecolor{babyblue}{rgb}{0.54, 0.81, 0.94}

\newtcolorbox[auto counter]{mybox}[1]{
  % Define a new tcolorbox style with custom paragraph spacing
  before upper={\parskip=10pt},
    after upper={\parskip=10pt},
    enhanced,
    arc= 1 mm,boxrule=1.5pt,
    colframe=babyblue!80!pink,
    colback=white,
    coltitle=black,
    % colback=blue!5!white,
    attach boxed title to top left=
    {xshift=1.5em,yshift=-\tcboxedtitleheight/2},
    boxed title style={size=small,
    % frame hidden,
    colback=White},
    top=0.15in,
    % fonttitle=\bfseries,
    title= {#1},
    breakable
  }

\newtcolorbox[auto counter]{Problem}[2][]{
    enhanced,drop shadow={Pink!50!white},
    colframe=pink!80!white,
    fonttitle=\bfseries,
    title=Problem ~\thetcbcounter. #2,
    %separator sign={.},
    coltitle=black,
    colback=pink!15,
    top=0.15in,
    breakable
  }

\newenvironment{solution}
  {\renewcommand\qedsymbol{$\blacksquare$}\begin{proof}[Solution]}
  {\end{proof}}

\theoremstyle{definition}
\newtheorem{definition}{Definition}[section]

%\theoremstyle{notation}
\newtheorem*{notation}{\underline{Notation}}
%\newtheorem*{convention}{\underline{Convention}}
\newtheorem*{convention}{\underline{Convention}}

\theoremstyle{remark}
\newtheorem*{remark}{Remark}

\newenvironment{amatrix}[2]{%% [2] for 2 parameters 
  \left[\begin{array}
    %{cc\,|\,cc}
    %  {@{}*{#2}{c}\,|\,c*{#1}{c}}
     {{}*{#1}{c}\,|\,c*{#2}{c}}
}{%
  \end{array}\right]
}
% For augmented matrix  
%https://tex.stackexchange.com/questions/2233/whats-the-best-way-make-an-augmented-coefficient-matrix


% defines the paragraph spacing
\setlength{\parskip}{0.5em}


\usepackage[sorting=none, citestyle=verbose-inote,backref=true,ibidtracker=context,mincrossrefs=99,backend=biber, 
url = false,
doi = false, isbn=false,]{biblatex}

\addbibresource{R10A21126.bib}

\usepackage{graphicx}
\graphicspath{ {images/} }
\usepackage{caption}

% global change
\SetKwInput{KwData}{Input}
\SetKwInput{KwResult}{Output}
% https://tex.stackexchange.com/questions/299771/how-do-i-rename-data-from-kwdata-and-result-from-kwresult-in-begi

\hypersetup{hidelinks}

\begin{document}


\begin{mybox}{Definition of \(\sigma\)-algebra}

  A nonempty collection \(\Sigma\) of subsets \(E\) is called a \(\sigma\)-algebra if it satisfies the following conditions \footcite[][49]{Wheeden_Zygmund_2015}:

  \begin{enumerate}[label=(\roman*)]
    \item \(\backslash E \in \Sigma\) if \(E \in \Sigma\)
    \item \(\cup_k E_k \in \Sigma\) if \(E_k \in \Sigma, k=1,2,\ldots\)
  \end{enumerate}
  

\end{mybox}
  

\begin{mybox}{Borel \(\sigma\)-algebra}
  The smallest \(\sigma\)-algebra of subsets of \(\mathbb{R}^n\) containing all the open subsets of
\(\mathbb{R}^n\) is called the Borel \(\sigma\)-algebra \(\mathfrak{B}\) of \(\mathbb{R}^n\), and the sets in \(\mathfrak{B}\) are called Borel subsets of \(\mathbb{R}^n\). Sets of type \(F_\sigma, G_\delta, F_{\sigma\delta}, G_{\delta\sigma}\) (see p.6 in Section 1.3), etc., are Borel sets.\footcite[][49]{Wheeden_Zygmund_2015}  


\end{mybox}


% Q1


  \begin{Problem}[]{Zygmund p59 exercise 08}

    Show that the Borel \(\sigma\)-algebra \(\mathfrak{B}\)  in \(\mathbb{R}^n\)  is the smallest \(\sigma\)-algebra containing the closed sets in \(\mathbb{R}^n\).
  \end{Problem}

  \begin{solution}\,

    To show that the Borel \(\sigma\)-algebra \(\mathfrak{B}\)  in \(\mathbb{R}^n\)  is the smallest \(\sigma\)-algebra containing the closed sets in \(\mathbb{R}^n\), we need to show that 

    \begin{enumerate}[label=(\roman*)]
      \item \(\mathfrak{B}\) contains the closed sets in \(\mathbb{R}^n\).
      \item \(\mathfrak{B} \subseteq A\) if \(A\) is a \(\sigma\)-algebra containing the closed sets in \(\mathbb{R}^n\).
    \end{enumerate}

(i)

For every closed subset \(E \in\mathbb{R}^n\), there is a open subset \(\backslash E \in \mathbb{R}^n\). By definition of the Borel \(\sigma\)-algebra \(\mathfrak{B}\), we have \(\backslash E \in \mathfrak{B}\), implying that \(E\in \mathfrak{B}\). Thus, \(\mathfrak{B}\) contains the closed sets in \(\mathbb{R}^n\).

(ii)

Let \(A\) be a \(\sigma\)-algebra containing the closed sets in \(\mathbb{R}^n\), the definition of \(\sigma\)-algebra implies that \(A\) is a \(\sigma\)-algebra containing all the open sets in \(\mathbb{R}^n\)

By definition of the Borel \(\sigma\)-algebra \(\mathfrak{B}\), it is the smallest \(\sigma\)-algebra containing the open sets in \(\mathbb{R}^n\). Thus, we have \(\mathfrak{B} \subseteq A\).

Therefore, we can conclude that the Borel \(\sigma\)-algebra \(\mathfrak{B}\)  in \(\mathbb{R}^n\)  is the smallest \(\sigma\)-algebra containing the closed sets in \(\mathbb{R}^n\).



\begin{equation*}
  \begin{aligned}
  \end{aligned}
\end{equation*}


  \end{solution}

  
  \begin{Problem}[]{Zygmund p59 exercise 09}
    If \(\{E_k\}_{k=1}^{\infty}\) is a sequence of sets with \(\sum_{k=1}^{\infty}|E_k|_e<+\infty\), show that \(\limsup_{k\to\infty}E_k\)
    (and so also \(\liminf_{k\to\infty}E_k\)) has measure zero.
  \end{Problem}

  \begin{mybox}{}

    
    \footcite[][49]{Wheeden_Zygmund_2015}Suppose \(\{E_k\}_{k=1}^{\infty}\) is a sequence of subsets:
    % \begin{enumerate}[label=(\roman*)]
    %   \item 
    %   \item \(\cup_k E_k \in \Sigma\) if \(E_k \in \Sigma, k=1,2,\ldots\)
    % \end{enumerate}
    \begin{equation*}
      \begin{aligned}
        \limsup_{k\to\infty}E_k := \lim_{n\to\infty} V_n = \bigcap_{n=1}^{\infty}  \bigcup_{k=n}^{\infty} E_k,\\
        \liminf_{k\to\infty}E_k := \lim_{n\to\infty} B_n = \bigcup_{n=1}^{\infty}  \bigcap_{k=n}^{\infty} E_k.
      \end{aligned}
    \end{equation*}
    In other words,
    \begin{equation*}
      \begin{aligned}
         V_n &= \bigcup_{k=n}^{\infty} E_k\searrow V= \bigcap_{n=1}^{\infty}  \bigcup_{k=n}^{\infty} E_k =: \limsup_{k\to\infty}E_k, \\
         B_n &= \bigcap_{k=n}^{\infty} E_k \nearrow B =  \bigcup_{n=1}^{\infty}  \bigcap_{k=n}^{\infty} E_k=:  \liminf_{k\to\infty}E_k.
      \end{aligned}
    \end{equation*}
    
    
  
  \end{mybox}


  \begin{solution}

  By the definition of the limit of a sequence, let \(\epsilon>0\), there exists \(N\in \mathbb{N}\) s.t. \(\forall n\geq N\),
  \begin{equation*}
    \begin{aligned}
      \sum_{k=n}^{\infty}|E_k|_e = \sum_{k=1}^{\infty}|E_k|_e - \sum_{k=1}^{n-1}|E_k|_e < \epsilon.
    \end{aligned}
  \end{equation*}
Since
  \begin{equation*}
    \begin{aligned}
      \limsup_{k\to\infty}E_k &:= \bigcap_{n=1}^{\infty}  \bigcup_{k=n}^{\infty} E_k\\
      &\subseteq \bigcup_{k=N}^{\infty} E_k,
    \end{aligned}
  \end{equation*}
  we have
  \begin{equation*}
    \begin{aligned}
      |\limsup_{k\to\infty}E_k|_e&\leq |\bigcup_{k=N}^{\infty} E_k|_e\\
      &=\sum_{k=N}^{\infty}|E_k|_e <\epsilon.
    \end{aligned}
  \end{equation*}

  Let \(\epsilon\to 0\), we have
  \begin{equation*}
    \begin{aligned}
      |\limsup_{k\to\infty}E_k|_e=0.
    \end{aligned}
  \end{equation*}

  \dotfill

  By the definition, we have
  \begin{equation*}
    \begin{aligned}
      \liminf_{k\to\infty}E_k \subseteq \limsup_{k\to\infty}E_k.
    \end{aligned}
  \end{equation*}
Thus, 
\begin{equation*}
  \begin{aligned}
    |\liminf_{k\to\infty}E_k|_e \leq |\limsup_{k\to\infty}E_k|_e = 0.
  \end{aligned}
\end{equation*}
Therefore, \( |\liminf_{k\to\infty}E_k|_e = 0\).


  \end{solution}


  \begin{Problem}[]{Zygmund p59 exercise 09}

    Show that there exist sets \(\{E_k\}_{k=1}^{\infty}\) such that \(E_k \searrow  E\), \(|E_k|_e < +\infty\), and \(\lim_{k\to\infty} |E_k|_e > |E|_e\) with strict inequality.
  \end{Problem}

  \begin{Problem}[]{Zygmund p76 exercise 02}
  Let \(f\) be a simple function, taking distinct values on disjoint sets \(E_1, \ldots, E_N\). Show that \(f\) is measurable if and only if \(E_i\) is measurable for \(1 \leq i \leq N\).

\end{Problem}


\begin{Problem}[]{Zygmund p76 exercise 05}
  Give an example to show that \(\phi(f(x))\) may not be measurable if \(\phi\) and \(f\) are measurable and finite.
  
  (Let \(F\) be the Cantor–Lebesgue function and let \(f\) be its inverse, suitably defined. Let \(\phi\) be the characteristic function of a set of measure zero whose image under \(F\) is not measurable.) 
  
  ---
  
  Show that the same may be true even if \(f\) is continuous. 
  
  (Let \(g(x) = x + F(x)\), where \(F\) is the Cantor–Lebesgue function, and consider \(f = g^{-1}\).) Cf. Exercise 22.

\end{Problem}


\end{document}