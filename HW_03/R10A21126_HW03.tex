\documentclass[UTF8,a4paper,10pt]{article}

\input{preamble.tex}

\begin{document}


\begin{mybox}{Definition of \(\sigma\)-algebra}

  A nonempty collection \(\Sigma\) of subsets \(E\) is called a \(\sigma\)-algebra if it satisfies the following conditions \footcite[][49]{Wheeden_Zygmund_2015}:
  \begin{enumerate}[label=(\roman*)]
    \item \(\backslash E \in \Sigma\) if \(E \in \Sigma\)
    \item \(\cup_k E_k \in \Sigma\) if \(E_k \in \Sigma, k=1,2,\ldots\)
  \end{enumerate}
  

\end{mybox}
  

\begin{mybox}{Borel \(\sigma\)-algebra}
  The smallest \(\sigma\)-algebra of subsets of \(\mathbb{R}^n\) containing all the open subsets of
\(\mathbb{R}^n\) is called the Borel \(\sigma\)-algebra \(\mathfrak{B}\) of \(\mathbb{R}^n\), and the sets in \(\mathfrak{B}\) are called Borel subsets of \(\mathbb{R}^n\). Sets of type \(F_\sigma, G_\delta, F_{\sigma\delta}, G_{\delta\sigma}\) (see p.6 in Section 1.3), etc., are Borel sets.\footcite[][49]{Wheeden_Zygmund_2015}  


\end{mybox}


% Q1


  \begin{Problem}[]{Zygmund p59 exercise 08}

    Show that the Borel \(\sigma\)-algebra \(\mathfrak{B}\)  in \(\mathbb{R}^n\)  is the smallest \(\sigma\)-algebra containing the closed sets in \(\mathbb{R}^n\).
  \end{Problem}

  \begin{solution}\,\\

    To show that the Borel \(\sigma\)-algebra \(\mathfrak{B}\)  in \(\mathbb{R}^n\)  is the smallest \(\sigma\)-algebra containing the closed sets in \(\mathbb{R}^n\), we need to show that 

    \begin{enumerate}[label=(\roman*)]
      \item \(\mathfrak{B}\) contains the closed sets in \(\mathbb{R}^n\).
      \item \(\mathfrak{B} \subseteq A\) if \(A\) is a \(\sigma\)-algebra containing the closed sets in \(\mathbb{R}^n\).
    \end{enumerate}

(i)

For every closed subset \(E \in\mathbb{R}^n\), there is a open subset \(\backslash E \in \mathbb{R}^n\). By definition of the Borel \(\sigma\)-algebra \(\mathfrak{B}\), we have \(\backslash E \in \mathfrak{B}\), implying that \(E\in \mathfrak{B}\). Thus, \(\mathfrak{B}\) contains the closed sets in \(\mathbb{R}^n\).

(ii)

Let \(A\) be a \(\sigma\)-algebra containing the closed sets in \(\mathbb{R}^n\), the definition of \(\sigma\)-algebra implies that \(A\) is a \(\sigma\)-algebra containing all the open sets in \(\mathbb{R}^n\)

By definition of the Borel \(\sigma\)-algebra \(\mathfrak{B}\), it is the smallest \(\sigma\)-algebra containing the closed sets in \(\mathbb{R}^n\). Thus, we have \(\mathfrak{B} \subseteq A\).

Therefore, we can conclude that the Borel \(\sigma\)-algebra \(\mathfrak{B}\)  in \(\mathbb{R}^n\)  is the smallest \(\sigma\)-algebra containing the closed sets in \(\mathbb{R}^n\).



\begin{equation*}
  \begin{aligned}
  \end{aligned}
\end{equation*}


  \end{solution}

  
  \begin{Problem}[]{Zygmund p59 exercise 09}
    If \(\{E_k\}_{k=1}^{\infty}\) is a sequence of sets with \(\sum_{k=1}^{\infty}|E_k|_e<+\infty\), show that \(\limsup_{k\to\infty}E_k\)
    (and so also \(\liminf_{k\to\infty}E_k\)) has measure zero.
  \end{Problem}

  \begin{mybox}{Definition of \(\sigma\)-algebra}

    
    \footcite[][49]{Wheeden_Zygmund_2015}Suppose \(\{E_k\}_{k=1}^{\infty}\) is a sequence of subsets:
    % \begin{enumerate}[label=(\roman*)]
    %   \item 
    %   \item \(\cup_k E_k \in \Sigma\) if \(E_k \in \Sigma, k=1,2,\ldots\)
    % \end{enumerate}
    \begin{equation*}
      \begin{aligned}
        \limsup_{k\to\infty}E_k := \lim_{n\to\infty} V_n = \bigcap_{n=1}^{\infty}  \bigcup_{k=n}^{\infty} E_k,\\
        \liminf_{k\to\infty}E_k := \lim_{n\to\infty} B_n = \bigcup_{n=1}^{\infty}  \bigcap_{k=n}^{\infty} E_k.
      \end{aligned}
    \end{equation*}
    In other words,
    \begin{equation*}
      \begin{aligned}
         V_n &= \bigcup_{k=n}^{\infty} E_k\searrow V= \bigcap_{n=1}^{\infty}  \bigcup_{k=n}^{\infty} E_k =: \limsup_{k\to\infty}E_k, \\
         B_n &= \bigcap_{k=n}^{\infty} E_k \nearrow B =  \bigcup_{n=1}^{\infty}  \bigcap_{k=n}^{\infty} E_k=:  \liminf_{k\to\infty}E_k.
      \end{aligned}
    \end{equation*}
    
    
  
  \end{mybox}


  \begin{solution}

  By the definition of the limit of a sequence, let \(\epsilon>0\), there exists \(N\in \mathbb{N}\) s.t. \(\forall n\geq N\),
  \begin{equation*}
    \begin{aligned}
      \sum_{k=n}^{\infty}|E_k|_e = \sum_{k=1}^{\infty}|E_k|_e - \sum_{k=1}^{n-1}|E_k|_e < \epsilon.
    \end{aligned}
  \end{equation*}
Since
  \begin{equation*}
    \begin{aligned}
      \limsup_{k\to\infty}E_k &:= \bigcap_{n=1}^{\infty}  \bigcup_{k=n}^{\infty} E_k\\
      &\subseteq \bigcup_{k=N}^{\infty} E_k,
    \end{aligned}
  \end{equation*}
  we have
  \begin{equation*}
    \begin{aligned}
      |\limsup_{k\to\infty}E_k|_e&\leq |\bigcup_{k=N}^{\infty} E_k|_e\\
      &=\sum_{k=N}^{\infty}|E_k|_e <\epsilon.
    \end{aligned}
  \end{equation*}

  Let \(\epsilon\to 0\), we have
  \begin{equation*}
    \begin{aligned}
      |\limsup_{k\to\infty}E_k|_e=0.
    \end{aligned}
  \end{equation*}

  \dotfill

  By the definition, we have
  \begin{equation*}
    \begin{aligned}
      \liminf_{k\to\infty}E_k \subseteq \limsup_{k\to\infty}E_k.
    \end{aligned}
  \end{equation*}
Thus, 
\begin{equation*}
  \begin{aligned}
    |\liminf_{k\to\infty}E_k|_e \leq |\limsup_{k\to\infty}E_k|_e = 0.
  \end{aligned}
\end{equation*}
Therefore, \( |\liminf_{k\to\infty}E_k|_e = 0\).


  \end{solution}


  \begin{Problem}[]{Zygmund p59 exercise 09}

    Show that there exist sets \(\{E_k\}_{k=1}^{\infty}\) such that \(E_k \searrow  E\), \(|E_k|_e < +\infty\), and \(\lim_{k\to\infty} |E_k|_e > |E|_e\) with strict inequality.
  \end{Problem}

\end{document}