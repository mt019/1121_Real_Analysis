\documentclass[UTF8,a4paper,10pt]{article}

% \begin{equation*}
%   \begin{aligned}
%   \end{aligned}
% \end{equation*}

% \begin{mybox}{}
% \end{mybox}


% \begin{Problem}[]{}
% \end{Problem}

% \begin{solution}\,
% \end{solution}
  

% \begin{enumerate}[label=(\alph*)]
% \end{enumerate} 

% \setcounter{section}{3} 
% \setcounter{theorem}{3}

% \begin{theorem}\label{thm:3.4}
%   If $E = \bigsqcup_{k} E_k$ is a countable union of sets, then $|E|_e \leq \sum_{k} |E_k|_e$.
%   \end{theorem}

%  \footcite[][42]{Wheeden_Zygmund_2015}

\input{preamble.tex}

\begin{document}




\begin{mybox}{}

  Let \(f\) be any measurable function defined on a set \(E\). If $f$ exists and is finite, we say that $f$ is Lebesgue integrable, or simply integrable, on $E$ and write $f \in L(E)$. Thus,
\[
L(E) = \left\{ f : \int_{E} f \text{ is finite} \right\}.
\]
\end{mybox}

\begin{mybox}{Theorem 5.5}

% \textbf{Theorem 5.5}
\begin{enumerate}[label=(\roman*)]
    \item If $f$ and $g$ are measurable and $0 \leq g \leq f$ on $E$, then $\int_E g \leq \int_E f $. In particular, $\int_E (\inf f) \leq \int_E f$.
    \item If $f$ is nonnegative and measurable on $E$ and $\int_E f$ is finite, then $f < +\infty$ a.e. in $E$.
    \item Let $E_1$ and $E_2$ be measurable and $E_1 \subset E_2$. If $f$ is nonnegative and measurable on $E_2$, then $\int_{E_1} f \leq \int_{E_2} f$.
\end{enumerate}

\textbf{Proof:} 

Parts (i) and (iii) follow from the relations $R(g, E) \subset R(f, E)$ and $R(f, E_1) \subset R(f, E_2)$, respectively. 

To prove (ii), we may assume that $|E| > 0$. If $f = +\infty$ in a subset $E_1$ of $E$ with positive measure, then by (iii) and (i), we have $\int_E f \geq \int_{E_1} f \geq \int_{E_1} a = a|E_1|$, no matter how large $a$ is. This contradicts the finiteness of $\int_E f$.
\end{mybox}


\begin{mybox}{Theorem 5.22}

  If $f \in L(E)$, then $f$ is finite a.e. in $E$.

\textbf{Proof:} If $f \in L(E)$, then $|f| \in L(E)$, and the result follows from Theorem 5.5(ii).

\end{mybox}


\begin{mybox}{Theorem 5.36 (Lebesgue's Dominated Convergence Theorem)}

  Let $\{f_k\}$ be a sequence of measurable functions on $E$ such that $f_k \to f$ a.e. in $E$. If there exists $\phi \in L(E)$ such that $|f_k| \leq \phi$ a.e. in $E$ for all $k$, then $\int_E f_k  \to \int_E f $.

\end{mybox}

\pagebreak

\begin{Problem}[]{Zygmund p109 exercise 04}

  
  If $f \in L(0, 1)$, show that $x^kf(x) \in L(0, 1)$ for $k = 1, 2, \ldots$, and that
\[
\int_{0}^{1} x^kf(x) \,dx \rightarrow 0.
\]

\end{Problem}

Let \(g_k(x)=x^kf(x)\) and \(E = (0,1)\). We have
\(g_k(x)\) measurable on \(E\), thus \(\int_{E} g_k\) exists.

For \( x\in (0,1)\), \(x_k \leq 1\), so \(g(x) = x^kf(x)\leq f(x), \forall k\in \mathbb{N} \).
Hence, 
\[\int_{E}g_k \leq \int_{E}f<\infty,\]
implying that \(g_k(x) = x^kf(x) \in L(0, 1)\). 

\dotfill

Since \(f\in L(E)\), \(f\) is finite a.e. in \(E\).

Besides, for all \(x\in E\), \(x^k\to 0\), as \(k\to\infty\).

Thus, \(g_k(x) = x^k f(x) \to 0\) a.e in \(E\). 
Additionally, \(|g_k| \leq |f|\), while \(f \in L(E)\). 

Therefore, by Theorem 5.36 (Lebesgue's Dominated Convergence Theorem), we have
\[\int_{E} g_k(x) \, dx \to \int_{E} 0 \, dx = 0.\]

\pagebreak

\begin{Problem}[]{Zygmund p109 exercise 05}
  
  Use Egorov's theorem to prove the bounded convergence theorem.

\end{Problem}

\begin{mybox}{}

  Given \(f\in L(E)\), \(\forall \epsilon > 0, \exists  \delta > 0,\) s.t. \(\forall F \subseteq E\) with \(|F|<\delta\), \(\int_{F}|f|<\epsilon\).

\end{mybox}

By Egorov's Theorem, 

given \(\epsilon>0\), find a closed subset \(F\) of \(E\) such that 
\begin{itemize}
  \item \(|E\setminus F|<\delta_\epsilon\), thus \(\int_{E\setminus F} f <\epsilon\);
  \item  and \(f_k\overset{u}{\to}f\) on \(F\), thus \(\int_F f_k \to \int_F f\) (By Uniform Convergence Theorem).
\end{itemize}

Additionally, since there is a finite constant $M$ such that $|f_k| \leq M$ a.e. in $E$, then $|f_k| \leq M$ a.e. in $E\setminus F$, implying that
\[
  \int_{E\setminus F} f_k \leq \int_{E\setminus F} M = M |E\setminus F| \leq M\delta_{\epsilon}.
\]

Hence,

\begin{align*}
  \int_{E} f - \int_{E} f_k &= \left(\int_{F} f+\int_{E\setminus F} f\right) -  \left(\int_{F} f_k + \int_{E\setminus F} f_k\right)\\
  &= \left(\int_{F} f-\int_{F} f_k \right)+\left(\int_{E\setminus F} f- \int_{E\setminus F} f_k \right)\\
  &\leq \left(\int_{F} f-\int_{F} f_k \right)+\left(\int_{E\setminus F} f + \int_{E\setminus F} f_k\right)
\end{align*}

It follows that:
\begin{align*}
  \lim_{k\to \infty}\left| \int_{E} f - \int_{E} f_k \right| 
  &\leq \lim_{k\to \infty}\left|\int_{F} f-\int_{F} f_k \right| +\left(\epsilon + M\delta_{\epsilon}\right)\\
  & = \epsilon + M\delta_{\epsilon}.
\end{align*}

Choose \(\delta_{\epsilon} \leq \frac{\epsilon}{M}\), then
\(\lim_{k\to \infty}\left| \int_{E} f - \int_{E} f_k \right| \leq 2\epsilon\). 

Letting \(\epsilon\to 0\), we can conclude that \(\int_{E} f \to \int_{E} f_k \).

\pagebreak

\begin{mybox}{Corollary 5.37 (Bounded Convergence Theorem)}

    Let $f_k$ be a sequence of measurable functions on $E$ such that $f_k \to f$ a.e. in $E$. If $|E| < +\infty$ and there is a finite constant $M$ such that $|f_k| \leq M$ a.e. in $E$, then
    \[
      \int_E f_k \,\to \int_E f .
    \]

\pagebreak

\end{mybox}

\begin{mybox}{Theorem 4.17 (Egorov's Theorem)}

  Suppose that $\{f_k\}$ is a sequence of measurable functions that converges almost everywhere in a set $E$ of finite measure to a finite limit $f$. Then, given $\varepsilon > 0$, there is a closed subset $F$ of $E$ such that $|E - F| < \varepsilon$ and $\{f_k\}$ converge uniformly to $f$ on $F$.

\end{mybox}

\begin{mybox}{Theorem 5.23}
\begin{enumerate}
  \item[(i)] If both $\int_E f$ and $\int_E g$ exist, and if $f \leq g$ a.e. in $E$, then $\int_E f \leq \int_E g$. Moreover, if $f$ and $g$ are functions with $f = g$ a.e. in $E$ and $\int_E f$ exists, then $\int_E g$ exists, and $\int_E f = \int_E g$.

  \item[(ii)] If $\int_{E_2} f$ exists and $E_1$ is a measurable subset of $E_2$, then $\int_{E_1} f$ exists.
\end{enumerate}
\end{mybox}

\begin{mybox}{Theorem 5.24}

  If $\int_E f$ exists and $E = \bigcup_k E_k$ is the countable union of disjoint measurable sets $E_k$, then
  \[
    \int_E f = \sum_k \int_{E_k} f.
  \]
  
  \end{mybox}

\begin{mybox}{Theorem 5.33 (Uniform Convergence Theorem)}
  Let $f_k \in L(E)$ for $k = 1, 2, \ldots$, and let $\{f_k\}$ converge uniformly to $f$ on $E$ where $|E| < +\infty$. Then $f \in L(E)$ and
  \[
    \int_E f_k \to \int_E f .
  \]

\end{mybox}





\begin{Problem}[]{Zygmund p109 exercise 06}



Let $f(x, y)$, $0 \leq x, y \leq 1$, satisfy the following conditions: for each $x$, $f(x, y)$ is an integrable function of $y$, and $\dfrac{\partial f(x, y)}{\partial x}$ is a bounded function of $(x, y)$. Show that $\dfrac{\partial f(x, y)}{\partial x}$ is a measurable function of $y$ for each $x$ and
\[
\frac{d}{dx} \int_{0}^{1} f(x, y) \,dy = \int_{0}^{1} \frac{\partial}{\partial x} f(x, y) \,dy.
\]

\end{Problem}


\begin{Problem}[]{Zygmund p109 exercise 09-10}
  \begin{itemize}
    \item   If $p > 0$ and $|f - f_k|^p \rightarrow 0$ as $k \rightarrow \infty$, show that $f_k \overset{m}{\longrightarrow} f$ on $E$ (and thus that there is a subsequence $f_{k_j} \rightarrow f$ a.e. in $E$).
    \item   If $p > 0$, $|f - f_k|^p \rightarrow 0$, and $|f_k|^p \leq M$ for all $k$, show that $|f|^p \leq M$.

  \end{itemize}

\end{Problem}


\end{document}