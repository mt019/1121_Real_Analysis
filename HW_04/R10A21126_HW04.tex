\documentclass[UTF8,a4paper,10pt]{article}

% \begin{equation*}
%   \begin{aligned}
%   \end{aligned}
% \end{equation*}

\input{preamble.tex}

\begin{document}


\begin{mybox}{}


\end{mybox}
  

  \begin{Problem}[]{
    % Zygmund p59 exercise 08
    }

    \begin{enumerate}[label=(\alph*)]
      \item     
      Suppose that \(\{E_k\}_{k=1}^{\infty}\) is a countable family of measurable subsets of \(\R^n\) and that
      \begin{equation*}
        \begin{aligned}
          \sum_{k=1}^{\infty}|E_k|<+\infty.
        \end{aligned}
      \end{equation*}

      Let \(E = \limsup_{k\to\infty}E_k\). Prove that \(|E| = 0\).
      \item 
      Given an irrational \(x\), one can show (using the pigeonhole principle, for example) that there exist infinitely many fractions \(\frac{p}{q}\), with relatively prime integers \(p\) and \(q\) such that 
      \begin{equation*}
        \begin{aligned}
          \bigg| x-\frac{p}{q} \bigg|\leq\frac{1}{q^2}.
        \end{aligned}
      \end{equation*}

    However, prove that the set of those \(x\in \R\) such that there exist infinitely many fractions \(\frac{p}{q}\), with relatively prime integers \(p\) and \(q\) such that 
    \(\forall\epsilon > 0\), 
    \begin{equation*}
      \begin{aligned}
        \bigg| x-\frac{p}{q} \bigg|\leq\frac{1}{q^{2+\epsilon}}
      \end{aligned}
    \end{equation*}
    is a set of measure zero.

    \end{enumerate}  
  \end{Problem}

  \begin{solution}\,



  \end{solution}

  % Q2

  \begin{Problem}[]{
    % Zygmund p59 exercise 09
    }
    \begin{enumerate}[label=(\alph*)]
      \item Let \(E\) be a subset of \(\R\) with \(|E|_e > 0\). Prove that for each \(0 < \alpha < 1\), there
      exists an open interval \(I\) so that
      \begin{equation*}
        \begin{aligned}
          |E \cap I|_e \geq \alpha |I|_e.
        \end{aligned}
      \end{equation*}
      Loosely speaking, this estimate shows that \(E\) contains almost a whole interval.
      \item Suppose \(E\) is a measurable subset of \(\R\) with \(|E| > 0\). Prove that the difference set of \(E\), which is defined by      
      \begin{equation*}
        \begin{aligned}
          |E - E| = \{x-y\in\R|x,y\in E\}
        \end{aligned}
      \end{equation*}
      contains an open interval centered at the origin.
    \end{enumerate}

  \end{Problem}


  \begin{solution}\,



  \end{solution}

\end{document}