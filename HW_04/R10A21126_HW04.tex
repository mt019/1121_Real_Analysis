\documentclass[UTF8,a4paper,10pt]{article}

% \begin{equation*}
%   \begin{aligned}
%   \end{aligned}
% \end{equation*}

% \documentclass[UTF8,a4paper,14pt]{article}
% \usepackage[utf8]{inputenc}
\usepackage{amsmath}
% \usepackage{algorithm,algorithmic}
\usepackage[linesnumbered,ruled,vlined]{algorithm2e}

% \usepackage{algorithmicx}
% \usepackage{algpseudocode}
\usepackage{hyperref}

% \usepackage{algpseudocode}
\usepackage{amssymb}
\usepackage{amsfonts}
%for 字體
%https://tug.org/FontCatalogue/
% \usepackage[T1]{fontenc}
% \usepackage{tgbonum}
% \usepackage[bitstream-charter]{mathdesign}
% \usepackage[T1]{fontenc}
% \usepackage{bm}#粗體
%\usepackage{boondox-calo}
\usepackage{textcomp}
\usepackage{fancyhdr}%导入fancyhdrf包
\usepackage{ctex}%导入ctex包
\usepackage{enumitem} %for在Latex使用條列式清單
\usepackage{varwidth}
\usepackage{soul} %for \ul
\usepackage{comment}%\begin{comment}\end{comment}
\usepackage{cancel}%\cancel{}
%\usepackage{unicode-math}

\usepackage[dvipsnames, svgnames, x_11names]{xcolor}

\usepackage[low-sup]{subdepth}
\usepackage{subdepth}

\newcommand{\indep}{\Perp \!\!\! \Perp}

\usepackage{amsthm}
\DeclareMathOperator{\E}{\mathbb{E}}
\DeclareMathOperator{\Var}{\textbf{Var}}
\DeclareMathOperator{\Cov}{\textbf{Cov}}
\DeclareMathOperator{\Cor}{\textbf{Cor}}
\DeclareMathOperator{\X}{\mathbf{X}}
\DeclareMathOperator{\Pro}{\mathbf{P}}
\DeclareMathOperator{\M}{\mathbf{M}}
\DeclareMathOperator{\Id}{\mathbf{I}}
\DeclareMathOperator{\Y}{\mathbf{Y}}
\DeclareMathOperator{\MSFE}{\mathbf{MSFE}}
\DeclareMathOperator{\e}{\mathbb{e}}
\DeclareMathOperator{\V}{\mathbf{V}} 
\DeclareMathOperator{\tr}{\text{tr}}
\DeclareMathOperator{\A}{\textbf{A}}
\DeclareMathOperator{\diag}{diag}
\DeclareRobustCommand{\rchi}{{\mathpalette\irchi\relax}}
\newcommand{\irchi}[2]{\raisebox{\depth}{$#1\chi$}} % inner command, used by \rchi
\DeclareMathOperator*{\argmax}{arg\,max}
\DeclareMathOperator*{\argmin}{arg\,min}

\DeclareMathSizes{20}{10}{10}{5}

\usepackage[a4paper, margin=1in]{geometry}
% \setlength\parskip{5ex}% it would be better define distance in ex (5ex) 
                         %  or in pt, pc, mm, etc (see edit below)

\setlength{\parindent}{0pt}
\usepackage{array, makecell} %


%中英文設定
%\usepackage{fontspec}
% \setmainfont{TeX Gyre Termes}
% \usepackage{xeCJK} %引用中文字的指令集
% %\setCJKmainfont{PMingLiU}
% \setCJKmainfont{DFKai-SB}





% \setmainfont{Times New Roman}
% \setCJKmonofont{DFKai-SB}
\pagenumbering{arabic}%设置页码格式
\pagestyle{fancy}
\fancyhead{} % 初始化页眉
\usepackage{advdate}

% \newcommand{\yesterday}{{\AdvanceDate[-1]\today}}

\fancyhead[C]{Real Analysis\quad HW 03\quad  R10A21126\quad  WANG YIFAN\quad   \today}
%\fancyhead[LE]{\textsl{\rightmark}}
%\fancyfoot{} % 初始化页脚
%\fancyfoot[LO]{奇数页左页脚}
%\fancyfoot[LE]{偶数页左页脚}
%\fancyfoot[RO]{奇数页右页脚}
%\fancyfoot[RE]{偶数页右页脚}

% \title{{Econometrics HW 05}}
% \author{R10A21126}
% \date{\today}

%\fancyhf{}
\usepackage{lastpage}
\cfoot{Page \thepage \hspace{1pt} of\, \pageref{LastPage}}

\renewcommand{\headrulewidth}{0.1pt}%分隔线宽度4磅
%\renewcommand{\footrulewidth}{4pt}

\allowdisplaybreaks
\usepackage[english]{babel}
%\usepackage{amsthm}
\newtheorem{theorem}{Theorem}[section]
\newtheorem{corollary}{Corollary}[theorem]
\newtheorem{lemma}[theorem]{Lemma}


\usepackage[most]{tcolorbox}

\definecolor{babyblue}{rgb}{0.54, 0.81, 0.94}

\newtcolorbox[auto counter]{mybox}[1]{
  % Define a new tcolorbox style with custom paragraph spacing
  before upper={\parskip=10pt},
    after upper={\parskip=10pt},
    enhanced,
    arc= 1 mm,boxrule=1.5pt,
    colframe=babyblue!80!pink,
    colback=white,
    coltitle=black,
    % colback=blue!5!white,
    attach boxed title to top left=
    {xshift=1.5em,yshift=-\tcboxedtitleheight/2},
    boxed title style={size=small,
    % frame hidden,
    colback=White},
    top=0.15in,
    % fonttitle=\bfseries,
    title= {#1},
    breakable
  }

\newtcolorbox[auto counter]{Problem}[2][]{
    enhanced,drop shadow={Pink!50!white},
    colframe=pink!80!white,
    fonttitle=\bfseries,
    title=Problem ~\thetcbcounter. #2,
    %separator sign={.},
    coltitle=black,
    colback=pink!15,
    top=0.15in,
    breakable
  }

\newenvironment{solution}
  {\renewcommand\qedsymbol{$\blacksquare$}\begin{proof}[Solution]}
  {\end{proof}}

\theoremstyle{definition}
\newtheorem{definition}{Definition}[section]

%\theoremstyle{notation}
\newtheorem*{notation}{\underline{Notation}}
%\newtheorem*{convention}{\underline{Convention}}
\newtheorem*{convention}{\underline{Convention}}

\theoremstyle{remark}
\newtheorem*{remark}{Remark}

\newenvironment{amatrix}[2]{%% [2] for 2 parameters 
  \left[\begin{array}
    %{cc\,|\,cc}
    %  {@{}*{#2}{c}\,|\,c*{#1}{c}}
     {{}*{#1}{c}\,|\,c*{#2}{c}}
}{%
  \end{array}\right]
}
% For augmented matrix  
%https://tex.stackexchange.com/questions/2233/whats-the-best-way-make-an-augmented-coefficient-matrix


% defines the paragraph spacing
\setlength{\parskip}{0.5em}


\usepackage[sorting=none, citestyle=verbose-inote,backref=true,ibidtracker=context,mincrossrefs=99,backend=biber, 
url = false,
doi = false, isbn=false,]{biblatex}

\addbibresource{R10A21126.bib}

\usepackage{graphicx}
\graphicspath{ {images/} }
\usepackage{caption}

% global change
\SetKwInput{KwData}{Input}
\SetKwInput{KwResult}{Output}
% https://tex.stackexchange.com/questions/299771/how-do-i-rename-data-from-kwdata-and-result-from-kwresult-in-begi

\hypersetup{hidelinks}

\begin{document}


% \begin{mybox}{}


% \end{mybox}
  

  \begin{Problem}[]{
    % Zygmund p59 exercise 08
    }

    \begin{enumerate}[label=(\alph*)]
      \item     
      Suppose that \(\{E_k\}_{k=1}^{\infty}\) is a countable family of measurable subsets of \(\R^n\) and that
      \begin{equation*}
        \begin{aligned}
          \sum_{k=1}^{\infty}|E_k|<+\infty.
        \end{aligned}
      \end{equation*}

      Let \(E = \limsup_{k\to\infty}E_k\). Prove that \(|E| = 0\).
      \item 
      Given an irrational \(x\), one can show (using the pigeonhole principle, for example) that there exist infinitely many fractions \(\frac{p}{q}\), with relatively prime integers \(p\) and \(q\) such that 
      \begin{equation*}
        \begin{aligned}
          \bigg| x-\frac{p}{q} \bigg|\leq\frac{1}{q^2}.
        \end{aligned}
      \end{equation*}

    However, prove that the set of those \(x\in \R\) such that there exist infinitely many fractions \(\frac{p}{q}\), with relatively prime integers \(p\) and \(q\) such that 
    \(\forall\epsilon > 0\), 
    \begin{equation*}
      \begin{aligned}
        \bigg| x-\frac{p}{q} \bigg|\leq\frac{1}{q^{2+\epsilon}}
      \end{aligned}
    \end{equation*}
    is a set of measure zero.

    \end{enumerate}  
  \end{Problem}

  \begin{solution}\,



  \end{solution}
  \pagebreak

  % Q2

  \begin{Problem}[]{
    % Zygmund p59 exercise 09
    }
    \begin{enumerate}[label=(\alph*)]
      \item Let \(E\) be a subset of \(\R\) with \(|E|_e > 0\). Prove that for each \(0 < \alpha < 1\), there
      exists an open interval \(I\) so that
      \begin{equation*}
        \begin{aligned}
          |E \cap I|_e \geq \alpha |I|_e.
        \end{aligned}
      \end{equation*}
      Loosely speaking, this estimate shows that \(E\) contains almost a whole interval.
      \item Suppose \(E\) is a measurable subset of \(\R\) with \(|E| > 0\). Prove that the difference set of \(E\), which is defined by      
      \begin{equation*}
        \begin{aligned}
          E - E = \{x-y\in\R|x,y\in E\}
        \end{aligned}
      \end{equation*}
      contains an open interval centered at the origin.
    \end{enumerate}

  \end{Problem}


  \begin{solution}\,

    (a)

    For any \(\alpha\in(0,1)\), let \(I\subseteq \R\) be an open set s.t. \(E\subseteq I\) and \(|E|_e \geq \alpha|I|_e\), implying that 
  
    \begin{equation*}
      \begin{aligned}
        \alpha |I|_e\leq |E|_e\leq |I|_e.
      \end{aligned}
    \end{equation*}

    Write the open set \(I\) as a countable union of disjoint open intervals:
    \begin{equation*}
      \begin{aligned}
        I = \bigsqcup_{k = 1}^{\infty}I_k.
      \end{aligned}
    \end{equation*}
    Thus, 
    \begin{equation*}
      \begin{aligned}
        E = E\bigcap I = E \bigcap\left(\bigsqcup_{k = 1}^{\infty}I_k\right) = \bigsqcup_{k = 1}^{\infty} \left(E\bigcap I_k\right).
      \end{aligned}
    \end{equation*}
    By the countable subadditivity of Lebesgue Outer Measure (Theorem \ref*{thm:3.4}), we have 
    \begin{equation*}
      \begin{aligned}
        |E|_e\leq \sum_{k = 1}^{\infty} \left|E\bigcap I_k\right|_e.
      \end{aligned}
    \end{equation*}

    Suppose, by way of contradiction, that \(\forall I_k\), 
    \begin{equation*}
      \begin{aligned}
         \left|E\bigcap I_k\right|_e <\alpha |I_k|_e,
      \end{aligned}
    \end{equation*}
    then we have
    \begin{equation*}
      \begin{aligned}
        |E|_e\leq \sum_{k = 1}^{\infty}\left|E\bigcap I_k\right|_e < \sum_{k = 1}^{\infty}\alpha |I_k|_e = \alpha \sum_{k = 1}^{\infty} |I_k|_e = \alpha |I|_e \leq |E|_e.
      \end{aligned}
    \end{equation*}
    The second equality holds since \(I_k\) are disjoint and open.

    Thus, it is implied that 
    \begin{equation*}
      \begin{aligned}
        |E|_e < |E|_e,
      \end{aligned}
    \end{equation*}

    which is a contradiction. Therefore, our assumption that \(\forall I_k, \left|E\bigcap I_k\right|_e <\alpha |I_k|_e\) must be false, which means that 
    there exists an open interval \(I\) so that
    \begin{equation*}
      \begin{aligned}
        |E \cap I|_e \geq \alpha |I|_e.
      \end{aligned}
    \end{equation*}

\dotfill

(b)

% \(D(E) = E-E = \{d|d = x-y, x\in E, y\in E\}\)

\(\exists G\) open s.t. \(E\subseteq G\) and \(|G| < |E|(1+\epsilon) \).

Since \(G\) is open, \(G\) can be written as a countable union of disjoint open intervals
\begin{equation*}
  \begin{aligned}
    G = \bigsqcup_{k = 1}^{\infty}\mathring{I}_k.
  \end{aligned}
\end{equation*}

Let \(E_k = \mathring{I}_k\cap E\), \(\{E_k\}_{k=1}^{\infty}\) is a sequence of disjoint measurable sets.

\begin{equation*}
  \begin{aligned}
    |G| = \sum_{k = 1}^{\infty}|\mathring{I}_k|,\\
    |E| = \sum_{k = 1}^{\infty}|E_k|.
  \end{aligned}
\end{equation*}
\begin{equation}
  \begin{aligned}
    |G| 
    &= \sum_{k = 1}^{\infty}|\mathring{I}_k|&&<|E|(1+\epsilon)\\
    &&&=\left(\sum_{k = 1}^{\infty}|E_k|\right)(1+\epsilon).\label{eq.01}
    % |E| = \sum_{k = 1}^{\infty}|E_k|.
  \end{aligned}
\end{equation}

\(\exists k_0\) s.t. \(|\mathring{I}_{k0}|<|E_{k0}|(1+\epsilon)\).

\begin{mybox}{}

  Suppose not; in other words, \(|\mathring{I}_{k}|\geq|E_{k}|(1+\epsilon), \forall k\),
  \begin{equation*}
    \begin{aligned}
  \sum_{k = 1}^{\infty}|\mathring{I}_k|\geq\left(\sum_{k = 1}^{\infty}|E_k|\right)(1+\epsilon),
    \end{aligned}
  \end{equation*}
    
    contradicting ~(\ref{eq.01}).

\end{mybox}


  Let \(\epsilon = \frac{1}{3}\), we have
  \begin{equation*}
    \begin{aligned}
      &|\mathring{I}_{k0}|<|E_{k0}|(1+\frac{1}{3}) = \frac{4}{3}|E_{k0}|,\\
      &\frac{3}{4}|\mathring{I}_{k0}|<|E_{k0}|.
    \end{aligned}
  \end{equation*}
Let \(E_{k0} + d = \{x+d \,|\, x\in E_{k0} \}\).

Claim: (to be proved by contradiction)

If \(|d| \leq \frac{1}{2}|\mathring{I}_{k0}|\),
\begin{equation*}
  \begin{aligned}
    (E_{k0} + d)\cap E_{k0} \neq \emptyset.
  \end{aligned}
\end{equation*}

\(\Rightarrow \)

\begin{equation*}
  \begin{aligned}
    \left(-\frac{1}{2}|\mathring{I}_{k0}|\,,\,\frac{1}{2}|\mathring{I}_{k0}|\right)&\subseteq\{x-y\,|\,x,y \in E_{k0}\}\\
    &\subseteq \{x-y\,|\,x,y \in E\}
  \end{aligned}
\end{equation*}


  
\end{solution}

\pagebreak

\begin{mybox}{Proof of Claim}

  Claim: 

If \(|d| \leq \frac{1}{2}|\mathring{I}_{k0}|\),
\begin{equation*}
  \begin{aligned}
    (E_{k0} + d)\cap E_{k0} \neq \emptyset.
  \end{aligned}
\end{equation*}

\begin{proof}
  Suppose not; in other words, \(E_{k0} + d \) and \(E_{k0}\) are disjoint measurable sets.
  \begin{equation*}
    \begin{aligned}
      |(E_{k0} + d)\cup E_{k0}| &= |(E_{k0} + d)| + |E_{k0}| \\
      &= 2|E_{k0}|.
    \end{aligned}
  \end{equation*}
  Since \(E_k = \mathring{I}_k\cap E\), we have
  \begin{equation*}
    \begin{aligned}
      (E_{k0} + d)\cup E_{k0} \subseteq (\mathring{I}_{k0} + d) \cup \mathring{I}_{k0},
    \end{aligned}
  \end{equation*}
  and
  \begin{equation*}
    \begin{aligned}
      |(E_{k0} + d)\cup E_{k0}| 
      &\leq |(\mathring{I}_{k0} + d) \cup \mathring{I}_{k0}|\\
      &= |\mathring{I}_{k0}| + |d|\\
      &< \frac{3}{2} |\mathring{I}_{k0}|.
    \end{aligned}
  \end{equation*}
  Thus, we have
  \begin{equation*}
    \begin{aligned}
      |(E_{k0} + d)\cup E_{k0}|  = 2|E_{k0}|
      &< \frac{3}{2} |\mathring{I}_{k0}| &&< \frac{3}{2} |E_{k0}|(1+\frac{1}{3})\\
      &&&=2|E_{k0}|.
    \end{aligned}
  \end{equation*}
  This leads to a contradiction: \(2|E_{k0}|<2|E_{k0}|\).

  Therefore, if \(|d| \leq \frac{1}{2}|\mathring{I}_{k0}|\),
  \begin{equation*}
    \begin{aligned}
      (E_{k0} + d)\cap E_{k0} \neq \emptyset.
    \end{aligned}
  \end{equation*}

\end{proof}


\end{mybox}



\begin{mybox}{Countable Subadditivity of Lebesgue Outer Measure}

  \setcounter{section}{3} 
  \setcounter{theorem}{3}
  
  \begin{theorem}\label{thm:3.4}
    If $E = \bigcup_{k} E_k$ is a countable union of sets, then $|E|_e \leq \sum_{k} |E_k|_e$.
    \end{theorem}
    
    \begin{proof}
    We may assume that $|E_k|_e < +\infty$ for each $k = 1, 2, \ldots$, since otherwise, the conclusion is obvious. Fix $\varepsilon > 0$. Given $k$, choose intervals $I^{(k)}_j$ such that $E_k \subset \bigcup_j I^{(k)}_j$ and $\sum_j v(I^{(k)}_j) < |E_k|_e + \varepsilon 2^{-k}$.
    
    Since $E \subset \bigcup_{j,k} I^{(k)}_j$, we have $|E|_e \leq \sum_{j,k} v(I^{(k)}_j) = \sum_k \sum_j v(I^{(k)}_j)$. Therefore,
    \[|E|_e \leq \sum_k (|E_k|_e + \varepsilon 2^{-k}) = \sum_k |E_k|_e + \varepsilon,\]
    and the result follows by letting $\varepsilon \to 0$.\footcite[][42]{Wheeden_Zygmund_2015}
    \end{proof}
     

\end{mybox}
  




\end{document}