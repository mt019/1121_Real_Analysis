\documentclass[UTF8,a4paper,10pt]{article}

% \begin{equation**}
%   \begin{aligned}
%   \end{aligned}
% \end{equation*}

% \begin{mybox}{}
% \end{mybox}


% \begin{Problem}[]{}
% \end{Problem}

% \begin{solution}\,
% \end{solution}
  

% \begin{enumerate}[label=(\alph*)]
% \end{enumerate} 

% \setcounter{section}{3} 
% \setcounter{theorem}{3}

% \begin{theorem}\label{thm:3.4}
%   If $E = \bigsqcup_{k} E_k$ is a countable union of sets, then $|E|_e \leq \sum_{k} |E_k|_e$.
%   \end{theorem}

%  \footcite[][42]{Wheeden_Zygmund_2015}

\input{preamble.tex}

\begin{document}

\begin{mybox}{Theorems 5.16}
  If $f_k$, $k = 1, 2, \ldots$, are nonnegative and measurable, then
\[
  \int_E \left(\sum_{k=1}^{\infty} f_k\right) = \sum_{k=1}^{\infty} \int_E f_k.
\]

\textbf{Proof.}
The functions $F_N$ defined by $F_N = \sum_{k=1}^{N} f_k$ are nonnegative and measurable, and increase to $\sum_{k=1}^{\infty} f_k$. Hence,
\[
  \int_E \left(\sum_{k=1}^{\infty} f_k\right)= \lim_{N \to \infty} \int_E  F_N = \lim_{N \to \infty}\sum_{k=1}^{N} f_k = \sum_{k=1}^{\infty} \int_E f_k.
\]
\end{mybox}

\begin{mybox}{Theorem 5.22}

If $f \in L(E)$, then $f$ is finite almost everywhere in $E$.

\end{mybox}

\begin{mybox}{}
  A series $\sum a_n$ is called absolutely convergent if $\sum |a_n|$ is convergent. 
  
  If $\sum a_n$ is convergent and $\sum |a_n|$ is divergent, we call the series conditionally convergent.

\end{mybox}

\begin{mybox}{The Monotone Convergence Criterion for Real Sequences (Real-Analysis-4th-Ed-Royden)}
  \textbf{Theorem 15:} A monotone sequence of real numbers converges if and only if it is bounded.

\end{mybox}

\begin{Problem}[]{Zygmund p109 exercise 13}
\begin{enumerate}[label=(\alph*)]
  \item Let $\{f_k\}$ be a sequence of measurable functions on $E$. Show that $\sum_{k} |f_k|$ converges absolutely a.e. in $E$ if $\sum_{k} \int_E |f_k| < +\infty$. (Use Theorems 5.16 and 5.22.)
  % \item If $\{r_k\}$ denotes the rational numbers in $[0,1]$ and $\{a_k\}$ satisfies $\int_{[0,1]} |a_k| \, d\lambda < +\infty$, show that $\int_{[0,1]} |a_k (x - r_k)|^{-1/2} \, d\lambda$ converges absolutely a.e. in $[0,1]$.

\end{enumerate}
\end{Problem}

Since 
\(|f_k|\) is nonnegative, and

\[\sum_{k} \int_E |f_k| < +\infty,\]

by Thm 5.16,

\[
  \int_E \left(\sum_{k=1}^{\infty} |f_k| \right) = \sum_{k=1}^{\infty} \int_E |f_k| <  +\infty,
\]

we have \(  \sum_{k=1}^{\infty} |f_k| \in L(E)\).


By Thm 5.22, \( \sum_{k=1}^{\infty} |f_k| \) is finite a.e in \(E\). 

Notice that \( \sum_{k=1}^{\infty} |f_k|\) is monotone increasing. 

We can conclude that \( \sum_{k=1}^{\infty} |f_k|\) converges a.e. in \(E\), by the Monotone Convergence Theorem for real sequences.






\pagebreak


\begin{mybox}{}
Given $\phi \geq 0$, let $L_{\phi}(E)$ denote the class of measurable functions $f$ such that $\phi(f) \in L(E)$. If $\phi(\alpha) = |\alpha|^p$, $0 < p < \infty$, the standard notation is
\[
L^p(E) = \left\{ f : \int_{E} |f|^p < +\infty \right\}, \quad 0 < p < \infty.
\]
Note that $L^1(E) = L(E)$. We will systematically study the $L^p$ classes in Chapter 8.

\end{mybox}


\begin{mybox}{Theorem 5.51}
  
If $0 < p < \infty$, $f \geq 0$, and $f \in L^p(E)$, then
\[
\int_{E} f^p = -\int_{0}^{\infty} \alpha^p \, d\omega(\alpha) = p \int_{0}^{\infty} \alpha^{p-1} \omega(\alpha) \, d\alpha,
\]
where the last integral may be interpreted as either a Lebesgue or an improper Riemann integral.

\end{mybox}

% \begin{Problem}[]{Zygmund p109 exercise 16}

%   Suppose that $f$ is nonnegative and measurable on $E$, and $\omega$ is finite on $(0, \infty)$. Show that Theorem 5.51 holds without any further restrictions (i.e., $f$ need not be in $L^p(E)$ and $|E|$ need not be finite) if we interpret
%   \[
%   \int_{0}^{\infty} \alpha^p \, d\omega(\alpha) = \lim_{{a \to 0^+}\atop{b \to \infty}} \int_{a}^{b} .
%   \]
%   For the first part, use the sets $E_{ab}$ to obtain the relation
%   \[
%   \int_{E} f^p = -\int_{0}^{\infty} \alpha^p \, d\omega(\alpha).
%   \]
%   If either $\int_{0}^{\infty} \alpha^p \, d\omega(\alpha)$ or $\int_{0}^{\infty} \alpha^{p-1} \omega(\alpha) \, d\alpha$ is finite, use Lemma 5.50 and the results of Exercises 14 or 15 to integrate by parts.
  
%   \end{Problem}


  \begin{Problem}[]{Zygmund p110 exercise 18}

    If $f \geq 0$, show that $f \in L^p$ if and only if $\sum_{k=-\infty}^{\infty} 2^{kp} \omega(2^k) < +\infty$. (Use Exercise 16.)


    \end{Problem}


    By Exercise 16, we have that if \(f\) is nonnegative and measurable on \(E\) and that \(\omega\) is finite on \((0,\infty)\), then 
    \[
      \int_{E} f^p = -\int_{0}^{\infty} \alpha^p \, d\omega(\alpha) = p \int_{0}^{\infty} \alpha^{p-1} \omega(\alpha) \, d\alpha.
      \]
    
      That is, it suffices to show that $f \in L^p$ if and only if
\[
\int_{0}^{\infty} \alpha^{p-1}\omega(\alpha) \,d\alpha <\infty.
\]


Now observe that
\[
\int_{0}^{\infty} \alpha^{p-1}\omega(\alpha) \,d\alpha =
\sum_{k=-\infty}^{\infty} \int_{2^k}^{2^{k+1}} \alpha^{p-1}\omega(\alpha)\,d\alpha.
\]

Since $\alpha^{p-1}$ is monotone increasing and $\omega(\alpha)$ is monotone decreasing, we have
\[
\int_{2^k}^{2^{k+1}} 2^{k(p-1)}\omega(2^{k+1}) \,d\alpha \leq
\int_{2^k}^{2^{k+1}} \alpha^{p-1}\omega(\alpha) \leq
\int_{2^k}^{2^{k+1}} 2^{(k+1)(p-1)}\omega(2^k).
\]
Simplifying, we get
\begin{align*}
  2^{k(p-1)+k}\omega(2^{k+1}) \,d\alpha &\leq
\int_{2^k}^{2^{k+1}} \alpha^{p-1}\omega(\alpha) \leq
2^{(k+1)(p-1)+k}\omega(2^k)\\
2^{(k+1)p-p}\omega(2^{k+1}) \,d\alpha &\leq
\int_{2^k}^{2^{k+1}} \alpha^{p-1}\omega(\alpha) \leq
2^{kp+(p-1)}\omega(2^k).
\end{align*}
Therefore,
\begin{align*}
  &2^{-p}\sum_{k = -\infty}^{\infty}2^{kp}\omega(2^{k})\quad\text{(Since \(k\) ranges from \(-\infty\) to \(\infty\)}\\ 
  =
  &2^{-p}\sum_{k = -\infty}^{\infty}2^{(k+1)p}\omega(2^{k+1}) 
  \leq\int_{0}^{\infty} \alpha^{p-1}\omega(\alpha) \,d\alpha =
  \sum_{k=-\infty}^{\infty} \int_{2^k}^{2^{k+1}} \alpha^{p-1}\omega(\alpha)\,d\alpha &\leq
 2^{p-1} \sum_{k=-\infty}^{\infty} 2^{kp}\omega(2k).
\end{align*}

So, $f \in L^p$ if and only if
\[
  \sum_{k=-\infty}^{\infty} 2^{kp} \omega(2^k) < +\infty.
\]


\end{document}