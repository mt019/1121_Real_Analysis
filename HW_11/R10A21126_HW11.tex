\documentclass[UTF8,a4paper,10pt]{article}

% \begin{equation**}
%   \begin{aligned}
%   \end{aligned}
% \end{equation*}

% \begin{mybox}{}
% \end{mybox}


% \begin{Problem}[]{}
% \end{Problem}

% \begin{solution}\,
% \end{solution}
  

% \begin{enumerate}[label=(\alph*)]
% \end{enumerate} 

% \setcounter{section}{3} 
% \setcounter{theorem}{3}

% \begin{theorem}\label{thm:3.4}
%   If $E = \bigsqcup_{k} E_k$ is a countable union of sets, then $|E|_e \leq \sum_{k} |E_k|_e$.
%   \end{theorem}

%  \footcite[][42]{Wheeden_Zygmund_2015}

% \documentclass[UTF8,a4paper,14pt]{article}
% \usepackage[utf8]{inputenc}
\usepackage{amsmath}
% \usepackage{algorithm,algorithmic}
\usepackage[linesnumbered,ruled,vlined]{algorithm2e}

% \usepackage{algorithmicx}
% \usepackage{algpseudocode}
\usepackage{hyperref}

% \usepackage{algpseudocode}
\usepackage{amssymb}
\usepackage{amsfonts}
%for 字體
%https://tug.org/FontCatalogue/
% \usepackage[T1]{fontenc}
% \usepackage{tgbonum}
% \usepackage[bitstream-charter]{mathdesign}
% \usepackage[T1]{fontenc}
% \usepackage{bm}#粗體
%\usepackage{boondox-calo}
\usepackage{textcomp}
\usepackage{fancyhdr}%导入fancyhdrf包
\usepackage{ctex}%导入ctex包
\usepackage{enumitem} %for在Latex使用條列式清單
\usepackage{varwidth}
\usepackage{soul} %for \ul
\usepackage{comment}%\begin{comment}\end{comment}
\usepackage{cancel}%\cancel{}
%\usepackage{unicode-math}

\usepackage[dvipsnames, svgnames, x_11names]{xcolor}

\usepackage[low-sup]{subdepth}
\usepackage{subdepth}

\newcommand{\indep}{\Perp \!\!\! \Perp}

\usepackage{amsthm}
\DeclareMathOperator{\E}{\mathbb{E}}
\DeclareMathOperator{\Var}{\textbf{Var}}
\DeclareMathOperator{\Cov}{\textbf{Cov}}
\DeclareMathOperator{\Cor}{\textbf{Cor}}
\DeclareMathOperator{\X}{\mathbf{X}}
\DeclareMathOperator{\Pro}{\mathbf{P}}
\DeclareMathOperator{\M}{\mathbf{M}}
\DeclareMathOperator{\Id}{\mathbf{I}}
\DeclareMathOperator{\Y}{\mathbf{Y}}
\DeclareMathOperator{\MSFE}{\mathbf{MSFE}}
\DeclareMathOperator{\e}{\mathbb{e}}
\DeclareMathOperator{\V}{\mathbf{V}} 
\DeclareMathOperator{\tr}{\text{tr}}
\DeclareMathOperator{\A}{\textbf{A}}
\DeclareMathOperator{\diag}{diag}
\DeclareRobustCommand{\rchi}{{\mathpalette\irchi\relax}}
\newcommand{\irchi}[2]{\raisebox{\depth}{$#1\chi$}} % inner command, used by \rchi
\DeclareMathOperator*{\argmax}{arg\,max}
\DeclareMathOperator*{\argmin}{arg\,min}

\DeclareMathSizes{20}{10}{10}{5}

\usepackage[a4paper, margin=1in]{geometry}
% \setlength\parskip{5ex}% it would be better define distance in ex (5ex) 
                         %  or in pt, pc, mm, etc (see edit below)

\setlength{\parindent}{0pt}
\usepackage{array, makecell} %


%中英文設定
%\usepackage{fontspec}
% \setmainfont{TeX Gyre Termes}
% \usepackage{xeCJK} %引用中文字的指令集
% %\setCJKmainfont{PMingLiU}
% \setCJKmainfont{DFKai-SB}





% \setmainfont{Times New Roman}
% \setCJKmonofont{DFKai-SB}
\pagenumbering{arabic}%设置页码格式
\pagestyle{fancy}
\fancyhead{} % 初始化页眉
\usepackage{advdate}

% \newcommand{\yesterday}{{\AdvanceDate[-1]\today}}

\fancyhead[C]{Real Analysis\quad HW 03\quad  R10A21126\quad  WANG YIFAN\quad   \today}
%\fancyhead[LE]{\textsl{\rightmark}}
%\fancyfoot{} % 初始化页脚
%\fancyfoot[LO]{奇数页左页脚}
%\fancyfoot[LE]{偶数页左页脚}
%\fancyfoot[RO]{奇数页右页脚}
%\fancyfoot[RE]{偶数页右页脚}

% \title{{Econometrics HW 05}}
% \author{R10A21126}
% \date{\today}

%\fancyhf{}
\usepackage{lastpage}
\cfoot{Page \thepage \hspace{1pt} of\, \pageref{LastPage}}

\renewcommand{\headrulewidth}{0.1pt}%分隔线宽度4磅
%\renewcommand{\footrulewidth}{4pt}

\allowdisplaybreaks
\usepackage[english]{babel}
%\usepackage{amsthm}
\newtheorem{theorem}{Theorem}[section]
\newtheorem{corollary}{Corollary}[theorem]
\newtheorem{lemma}[theorem]{Lemma}


\usepackage[most]{tcolorbox}

\definecolor{babyblue}{rgb}{0.54, 0.81, 0.94}

\newtcolorbox[auto counter]{mybox}[1]{
  % Define a new tcolorbox style with custom paragraph spacing
  before upper={\parskip=10pt},
    after upper={\parskip=10pt},
    enhanced,
    arc= 1 mm,boxrule=1.5pt,
    colframe=babyblue!80!pink,
    colback=white,
    coltitle=black,
    % colback=blue!5!white,
    attach boxed title to top left=
    {xshift=1.5em,yshift=-\tcboxedtitleheight/2},
    boxed title style={size=small,
    % frame hidden,
    colback=White},
    top=0.15in,
    % fonttitle=\bfseries,
    title= {#1},
    breakable
  }

\newtcolorbox[auto counter]{Problem}[2][]{
    enhanced,drop shadow={Pink!50!white},
    colframe=pink!80!white,
    fonttitle=\bfseries,
    title=Problem ~\thetcbcounter. #2,
    %separator sign={.},
    coltitle=black,
    colback=pink!15,
    top=0.15in,
    breakable
  }

\newenvironment{solution}
  {\renewcommand\qedsymbol{$\blacksquare$}\begin{proof}[Solution]}
  {\end{proof}}

\theoremstyle{definition}
\newtheorem{definition}{Definition}[section]

%\theoremstyle{notation}
\newtheorem*{notation}{\underline{Notation}}
%\newtheorem*{convention}{\underline{Convention}}
\newtheorem*{convention}{\underline{Convention}}

\theoremstyle{remark}
\newtheorem*{remark}{Remark}

\newenvironment{amatrix}[2]{%% [2] for 2 parameters 
  \left[\begin{array}
    %{cc\,|\,cc}
    %  {@{}*{#2}{c}\,|\,c*{#1}{c}}
     {{}*{#1}{c}\,|\,c*{#2}{c}}
}{%
  \end{array}\right]
}
% For augmented matrix  
%https://tex.stackexchange.com/questions/2233/whats-the-best-way-make-an-augmented-coefficient-matrix


% defines the paragraph spacing
\setlength{\parskip}{0.5em}


\usepackage[sorting=none, citestyle=verbose-inote,backref=true,ibidtracker=context,mincrossrefs=99,backend=biber, 
url = false,
doi = false, isbn=false,]{biblatex}

\addbibresource{R10A21126.bib}

\usepackage{graphicx}
\graphicspath{ {images/} }
\usepackage{caption}

% global change
\SetKwInput{KwData}{Input}
\SetKwInput{KwResult}{Output}
% https://tex.stackexchange.com/questions/299771/how-do-i-rename-data-from-kwdata-and-result-from-kwresult-in-begi

\hypersetup{hidelinks}

\begin{document}

\begin{mybox}{Theorems 5.16}
  If $f_k$, $k = 1, 2, \ldots$, are nonnegative and measurable, then
\[
  \int_E \left(\sum_{k=1}^{\infty} f_k\right) = \sum_{k=1}^{\infty} \int_E f_k.
\]

\textbf{Proof.}
The functions $F_N$ defined by $F_N = \sum_{k=1}^{N} f_k$ are nonnegative and measurable, and increase to $\sum_{k=1}^{\infty} f_k$. Hence,
\[
  \int_E \left(\sum_{k=1}^{\infty} f_k\right)= \lim_{N \to \infty} \int_E  F_N = \lim_{N \to \infty}\sum_{k=1}^{N} f_k = \sum_{k=1}^{\infty} \int_E f_k.
\]
\end{mybox}

\begin{mybox}{Theorem 5.22}

If $f \in L(E)$, then $f$ is finite almost everywhere in $E$.

\end{mybox}

\begin{mybox}{}
  A series $\sum a_n$ is called absolutely convergent if $\sum |a_n|$ is convergent. 
  
  If $\sum a_n$ is convergent and $\sum |a_n|$ is divergent, we call the series conditionally convergent.

\end{mybox}

\begin{mybox}{The Monotone Convergence Criterion for Real Sequences (Real-Analysis-4th-Ed-Royden)}
  \textbf{Theorem 15:} A monotone sequence of real numbers converges if and only if it is bounded.

\end{mybox}

\begin{Problem}[]{Zygmund p109 exercise 13}
\begin{enumerate}[label=(\alph*)]
  \item Let $\{f_k\}$ be a sequence of measurable functions on $E$. Show that $\sum_{k} |f_k|$ converges absolutely a.e. in $E$ if $\sum_{k} \int_E |f_k| < +\infty$. (Use Theorems 5.16 and 5.22.)
  % \item If $\{r_k\}$ denotes the rational numbers in $[0,1]$ and $\{a_k\}$ satisfies $\int_{[0,1]} |a_k| \, d\lambda < +\infty$, show that $\int_{[0,1]} |a_k (x - r_k)|^{-1/2} \, d\lambda$ converges absolutely a.e. in $[0,1]$.

\end{enumerate}
\end{Problem}

Since 
\(|f_k|\) is nonnegative, and

\[\sum_{k} \int_E |f_k| < +\infty,\]

by Thm 5.16,

\[
  \int_E \left(\sum_{k=1}^{\infty} |f_k| \right) = \sum_{k=1}^{\infty} \int_E |f_k| <  +\infty,
\]

implying that \(  \sum_{k=1}^{\infty} |f_k| \in L(E)\).


By Thm 5.22, \( \sum_{k=1}^{\infty} |f_k| \) is finite a.e in \(E\). Notice that \( \sum_{k=1}^{\infty} |f_k|\) is monotone increasing. We can conclude that \( \sum_{k=1}^{\infty} |f_k|\) converges a.e. in \(E\), by the Monotone Convergence Theorem for real sequences.






\pagebreak


\begin{mybox}{}
Given $\phi \geq 0$, let $L_{\phi}(E)$ denote the class of measurable functions $f$ such that $\phi(f) \in L(E)$. If $\phi(\alpha) = |\alpha|^p$, $0 < p < \infty$, the standard notation is
\[
L^p(E) = \left\{ f : \int_{E} |f|^p < +\infty \right\}, \quad 0 < p < \infty.
\]
Note that $L^1(E) = L(E)$. We will systematically study the $L^p$ classes in Chapter 8.

\end{mybox}


\begin{mybox}{Theorem 5.51}
  
If $0 < p < \infty$, $f \geq 0$, and $f \in L^p(E)$, then
\[
\int_{E} f^p = -\int_{0}^{\infty} \alpha^p \, d\omega(\alpha) = p \int_{0}^{\infty} \alpha^{p-1} \omega(\alpha) \, d\alpha,
\]
where the last integral may be interpreted as either a Lebesgue or an improper Riemann integral.

\end{mybox}

\begin{Problem}[]{Zygmund p109 exercise 16}

  Suppose that $f$ is nonnegative and measurable on $E$, and $\omega$ is finite on $(0, \infty)$. Show that Theorem 5.51 holds without any further restrictions (i.e., $f$ need not be in $L^p(E)$ and $|E|$ need not be finite) if we interpret
  \[
  \int_{0}^{\infty} \alpha^p \, d\omega(\alpha) = \lim_{{a \to 0^+}\atop{b \to \infty}} \int_{a}^{b} .
  \]
  For the first part, use the sets $E_{ab}$ to obtain the relation
  \[
  \int_{E} f^p = -\int_{0}^{\infty} \alpha^p \, d\omega(\alpha).
  \]
  If either $\int_{0}^{\infty} \alpha^p \, d\omega(\alpha)$ or $\int_{0}^{\infty} \alpha^{p-1} \omega(\alpha) \, d\alpha$ is finite, use Lemma 5.50 and the results of Exercises 14 or 15 to integrate by parts.
  
  \end{Problem}


  \begin{Problem}[]{Zygmund p110 exercise 18}

    If $f \geq 0$, show that $f \in L^p$ if and only if $\sum_{k=-\infty}^{\infty} 2^{kp} \omega(2^k) < +\infty$. (Use Exercise 16.)


    \end{Problem}
  

\end{document}