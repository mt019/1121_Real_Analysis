\documentclass[UTF8,a4paper,10pt]{article}

% \begin{equation*}
%   \begin{aligned}
%   \end{aligned}
% \end{equation*}

% \begin{mybox}{}
% \end{mybox}


% \begin{Problem}[]{}
% \end{Problem}

% \begin{solution}\,
% \end{solution}
  

% \begin{enumerate}[label=(\alph*)]
% \end{enumerate} 

% \setcounter{section}{3} 
% \setcounter{theorem}{3}

% \begin{theorem}\label{thm:3.4}
%   If $E = \bigsqcup_{k} E_k$ is a countable union of sets, then $|E|_e \leq \sum_{k} |E_k|_e$.
%   \end{theorem}

%  \footcite[][42]{Wheeden_Zygmund_2015}

% \documentclass[UTF8,a4paper,14pt]{article}
% \usepackage[utf8]{inputenc}
\usepackage{amsmath}
% \usepackage{algorithm,algorithmic}
\usepackage[linesnumbered,ruled,vlined]{algorithm2e}

% \usepackage{algorithmicx}
% \usepackage{algpseudocode}
\usepackage{hyperref}

% \usepackage{algpseudocode}
\usepackage{amssymb}
\usepackage{amsfonts}
%for 字體
%https://tug.org/FontCatalogue/
% \usepackage[T1]{fontenc}
% \usepackage{tgbonum}
% \usepackage[bitstream-charter]{mathdesign}
% \usepackage[T1]{fontenc}
% \usepackage{bm}#粗體
%\usepackage{boondox-calo}
\usepackage{textcomp}
\usepackage{fancyhdr}%导入fancyhdrf包
\usepackage{ctex}%导入ctex包
\usepackage{enumitem} %for在Latex使用條列式清單
\usepackage{varwidth}
\usepackage{soul} %for \ul
\usepackage{comment}%\begin{comment}\end{comment}
\usepackage{cancel}%\cancel{}
%\usepackage{unicode-math}

\usepackage[dvipsnames, svgnames, x_11names]{xcolor}

\usepackage[low-sup]{subdepth}
\usepackage{subdepth}

\newcommand{\indep}{\Perp \!\!\! \Perp}

\usepackage{amsthm}
\DeclareMathOperator{\E}{\mathbb{E}}
\DeclareMathOperator{\Var}{\textbf{Var}}
\DeclareMathOperator{\Cov}{\textbf{Cov}}
\DeclareMathOperator{\Cor}{\textbf{Cor}}
\DeclareMathOperator{\X}{\mathbf{X}}
\DeclareMathOperator{\Pro}{\mathbf{P}}
\DeclareMathOperator{\M}{\mathbf{M}}
\DeclareMathOperator{\Id}{\mathbf{I}}
\DeclareMathOperator{\Y}{\mathbf{Y}}
\DeclareMathOperator{\MSFE}{\mathbf{MSFE}}
\DeclareMathOperator{\e}{\mathbb{e}}
\DeclareMathOperator{\V}{\mathbf{V}} 
\DeclareMathOperator{\tr}{\text{tr}}
\DeclareMathOperator{\A}{\textbf{A}}
\DeclareMathOperator{\diag}{diag}
\DeclareRobustCommand{\rchi}{{\mathpalette\irchi\relax}}
\newcommand{\irchi}[2]{\raisebox{\depth}{$#1\chi$}} % inner command, used by \rchi
\DeclareMathOperator*{\argmax}{arg\,max}
\DeclareMathOperator*{\argmin}{arg\,min}

\DeclareMathSizes{20}{10}{10}{5}

\usepackage[a4paper, margin=1in]{geometry}
% \setlength\parskip{5ex}% it would be better define distance in ex (5ex) 
                         %  or in pt, pc, mm, etc (see edit below)

\setlength{\parindent}{0pt}
\usepackage{array, makecell} %


%中英文設定
%\usepackage{fontspec}
% \setmainfont{TeX Gyre Termes}
% \usepackage{xeCJK} %引用中文字的指令集
% %\setCJKmainfont{PMingLiU}
% \setCJKmainfont{DFKai-SB}





% \setmainfont{Times New Roman}
% \setCJKmonofont{DFKai-SB}
\pagenumbering{arabic}%设置页码格式
\pagestyle{fancy}
\fancyhead{} % 初始化页眉
\usepackage{advdate}

% \newcommand{\yesterday}{{\AdvanceDate[-1]\today}}

\fancyhead[C]{Real Analysis\quad HW 03\quad  R10A21126\quad  WANG YIFAN\quad   \today}
%\fancyhead[LE]{\textsl{\rightmark}}
%\fancyfoot{} % 初始化页脚
%\fancyfoot[LO]{奇数页左页脚}
%\fancyfoot[LE]{偶数页左页脚}
%\fancyfoot[RO]{奇数页右页脚}
%\fancyfoot[RE]{偶数页右页脚}

% \title{{Econometrics HW 05}}
% \author{R10A21126}
% \date{\today}

%\fancyhf{}
\usepackage{lastpage}
\cfoot{Page \thepage \hspace{1pt} of\, \pageref{LastPage}}

\renewcommand{\headrulewidth}{0.1pt}%分隔线宽度4磅
%\renewcommand{\footrulewidth}{4pt}

\allowdisplaybreaks
\usepackage[english]{babel}
%\usepackage{amsthm}
\newtheorem{theorem}{Theorem}[section]
\newtheorem{corollary}{Corollary}[theorem]
\newtheorem{lemma}[theorem]{Lemma}


\usepackage[most]{tcolorbox}

\definecolor{babyblue}{rgb}{0.54, 0.81, 0.94}

\newtcolorbox[auto counter]{mybox}[1]{
  % Define a new tcolorbox style with custom paragraph spacing
  before upper={\parskip=10pt},
    after upper={\parskip=10pt},
    enhanced,
    arc= 1 mm,boxrule=1.5pt,
    colframe=babyblue!80!pink,
    colback=white,
    coltitle=black,
    % colback=blue!5!white,
    attach boxed title to top left=
    {xshift=1.5em,yshift=-\tcboxedtitleheight/2},
    boxed title style={size=small,
    % frame hidden,
    colback=White},
    top=0.15in,
    % fonttitle=\bfseries,
    title= {#1},
    breakable
  }

\newtcolorbox[auto counter]{Problem}[2][]{
    enhanced,drop shadow={Pink!50!white},
    colframe=pink!80!white,
    fonttitle=\bfseries,
    title=Problem ~\thetcbcounter. #2,
    %separator sign={.},
    coltitle=black,
    colback=pink!15,
    top=0.15in,
    breakable
  }

\newenvironment{solution}
  {\renewcommand\qedsymbol{$\blacksquare$}\begin{proof}[Solution]}
  {\end{proof}}

\theoremstyle{definition}
\newtheorem{definition}{Definition}[section]

%\theoremstyle{notation}
\newtheorem*{notation}{\underline{Notation}}
%\newtheorem*{convention}{\underline{Convention}}
\newtheorem*{convention}{\underline{Convention}}

\theoremstyle{remark}
\newtheorem*{remark}{Remark}

\newenvironment{amatrix}[2]{%% [2] for 2 parameters 
  \left[\begin{array}
    %{cc\,|\,cc}
    %  {@{}*{#2}{c}\,|\,c*{#1}{c}}
     {{}*{#1}{c}\,|\,c*{#2}{c}}
}{%
  \end{array}\right]
}
% For augmented matrix  
%https://tex.stackexchange.com/questions/2233/whats-the-best-way-make-an-augmented-coefficient-matrix


% defines the paragraph spacing
\setlength{\parskip}{0.5em}


\usepackage[sorting=none, citestyle=verbose-inote,backref=true,ibidtracker=context,mincrossrefs=99,backend=biber, 
url = false,
doi = false, isbn=false,]{biblatex}

\addbibresource{R10A21126.bib}

\usepackage{graphicx}
\graphicspath{ {images/} }
\usepackage{caption}

% global change
\SetKwInput{KwData}{Input}
\SetKwInput{KwResult}{Output}
% https://tex.stackexchange.com/questions/299771/how-do-i-rename-data-from-kwdata-and-result-from-kwresult-in-begi

\hypersetup{hidelinks}

\begin{document}




\begin{mybox}{4.2 Semicontinuous Functions}

 \footcite[][69]{Wheeden_Zygmund_2015}
Let $f$ be defined on $E$, and let $x_0$ be a limit point of $E$ that lies in $E$. Then $f$ is said to be upper semicontinuous at $x_0$ if


\[
\limsup_{\substack{x\to x_0 \\ x\in E}} f(x) \leq f(x_0).
\]

We will usually abbreviate this by saying that $f$ is usc at $x_0$. Note that if $f(x_0) = +\infty$, then $f$ is automatically usc at $x_0$; otherwise, the statement that $f$ is usc at $x_0$ means that given $M > f(x_0)$, there exists $\delta > 0$ such that $f(x) < M$ for all $x \in E$ that lie in the ball $|x - x_0| < \delta$. Intuitively, this means that near $x_0$, the values of $f$ do not exceed $f(x_0)$ by a fixed amount.

Similarly, $f$ is said to be lower semicontinuous at $x_0$, or lsc at $x_0$, if

\[
\liminf_{\substack{x\to x_0 \\ x\in E}} f(x) \geq f(x_0).
\]


Thus, if $f(x_0) = -\infty$, $f$ is lsc at $x_0$, while if $f(x_0) > -\infty$, the definition amounts to saying that given $m < f(x_0)$, there exists $\delta > 0$ such that $f(x) > m$ if $x \in E$ and $|x - x_0| < \delta$. Equivalently, $f$ is lsc at $x_0$ if and only if $-f$ is usc at $x_0$.

It follows that $f$ is continuous at $x_0$ if and only if $|f(x_0)| < +\infty$ and $f$ is both usc and lsc at $x_0$.

\end{mybox}


\begin{Problem}[]{Zygmund p77 exercise 11}

  Let $f$ be defined on $\mathbb{R}^n$, and let $B(x)$ denote the open ball $\{y : |x - y| < r\}$ with center $x$ and fixed radius $r$. Show that the function $g(x) = \sup\{f(y) : y \in B(x)\}$ is lsc (lower semi-continuous), and that the function $h(x) = \inf\{f(y) : y \in B(x)\}$ is usc (upper semi-continuous) on $\mathbb{R}^n$. Is the same true for the closed ball $\{y : |x - y| \leq r\}$?


\end{Problem}

(a) 
lsc


To prove by contradiction, we assume that \(g(x) = \sup\{f(y) : y \in B(x)\}\) is not lsc, to be specific,  \(\exists x_0\in \R^n\)
 
\[
\liminf_{\substack{x\to x_0 \\ x\in \R^n}} g(x) < g(x_0),
\]

that is,

\[
\liminf_{\substack{x\to x_0 \\ x\in \R^n}} \sup\{f(y) : y \in B(x)\} < \sup\{f(y) : y \in B(x_0)\}.
\]

Thus, by "lim inf", \(\exists x_n\to x_0\), \(\exists \epsilon>0\) s.t. 

\[
\sup\{f(y) : y \in B(x_n)\} < \sup\{f(y) : y \in B(x_0)\}-\epsilon.
\]

By "sup", \(\exists y\in B(x_0)\), s.t. 

\begin{align*}
  f(y) &> \sup\{f(y) : y \in B(x_0)\}-\epsilon/2\\
  &> \sup\{f(y) : y \in B(x_n)\}+\epsilon/2.
\end{align*}

Note that \(\exists n0\) s.t. \(|x_{n0}-x_0|< r-|y-x_0|\), we have

\begin{align*}
  |y - x_{n0}| < |y - x_0| + |x_{n0} - x_0| < r,
\end{align*}

implying that \(y\in B(x_{n_0})\).

Thus, 

\begin{align*}
  f(y) &> \sup \{f(z) : z\in B(x_{n_0})\} +\epsilon/2\\
  &> f(y) +\epsilon/2,
\end{align*}

which is a contradiction.

Therefore, we can conclude that \(g(x)\) is lower semicontinuous on \(\R^n\).

\dotfill

(b) 
usc:

Since \(f(x)\) is lsc if and only if \(-f(x)\) is usc, to show that \(h(x)\) is usc is to show that \(-h(x)\) is usc.


\begin{align*}
  -h(x) = 
  -\inf\{f(y):y\in B(x)\} 
  &= \sup\{-f(y):y\in B(x)\} \\
  &= \sup\{\phi(y):y\in B(x)\}
\end{align*}

which is lsc by (1).

Thus, we can see that \(h(x)\) is usc on \(\R^n\).

\dotfill

(c)

For the closed ball $\bar{B}  = \{y : |x - y| \leq r\}$, the same properties hold.

\pagebreak

\begin{Problem}[]{Zygmund p77 exercise 12}
  If $f(x)$, $x \in \mathbb{R}^1$, is continuous at almost every point of an interval $[a, b]$, show that $f$ is measurable on $[a, b]$. 
  
  ---

  Generalize this to functions defined in $\mathbb{R}^n$. 
  
  For a constructive proof, use the subintervals of a sequence of partitions to define a sequence of simple measurable functions converging to $f$ a.e. in $[a, b]$. Use Theorem 4.12. See also the proof of Theorem 5.54.

    \end{Problem}

    Since $f(x)$ is continuous at almost every point of $[a, b]$,
    
    there exists \(Z\subset [a,b]\) and s.t. $|Z| = 0$ and \(f\) is continuous on \(E = [a,b]\setminus Z\).


Note that $Z$ is measurable, and
$E = [a, b] \setminus Z$ is also measurable.


For any finite $\alpha$, we have

\[\{x \in [a, b] : f(x) > \alpha\} = \{x \in E  : f(x) > \alpha\} \sqcup \{x \in Z  : f(x) > \alpha\}.\]    


Note that $\{x \in E : f(x) > \alpha\}$ is measurable since $f$ is continuous, thus
measurable on $E$.


Since $\{x \in Z : f(x) > \alpha\} \subseteq Z$, thus $|\{x \in Z : f(x) > \alpha\}| = 0$, implying that
$\{x \in Z : f(x) > \alpha\}$ is also measurable.
Thus $\{x \in [a, b] : f(x) > \alpha\}$ is measurable. This means $f$ is measurable
on $[a, b]$.

  
\pagebreak

  \begin{Problem}[]{Zygmund p78 exercise 15}
Let $\{f_k\}$ be a sequence of measurable functions defined on a measurable set $E$ with $|E| < +\infty$. If $|f_k(x)| \leq M_x < +\infty$ for all $k$ for each $x \in E$, show that given $\varepsilon > 0$, there is a closed set $F \subset E$ and a finite $M$ such that $|E - F| < \varepsilon$ and $|f_k(x)| \leq M$ for all $k$ and all $x \in F$.

  \end{Problem}

  Let 
  
  \[ E_m := \left\{ |f_k(x)| \leq m , \forall  k\in \mathbb{N} \right\} = \bigcap_{k=1}^{\infty} \left\{ |f_k(x)| \leq m\right\}.\]

  Notice that \(E_M\) being measurable implies that its complement \(E_M^c\) is measurable (By Thm 3.17).

  Since \(E_m\nearrow \), and \(\forall x\in E, \exists M_x\) s.t 

  \[x\in \{|f_k(x)|\leq M_x\, \forall k \in \mathbb{N} \} = E_{M_x}\]
  
  We have \(E_m\to E\). Hence, by Thm 3.26, we have \(\lim_{m\to\infty}|E_m| = |E| < +\infty\).



  Thus, \(\forall \epsilon > 0\), there exists \(M < \infty\), s.t. \(\{|f_k(x)|\leq M\}\) for all k and all \(x\in E_M\), and

  \[ |E| - |E_M| = |E \setminus E_M| < \epsilon/2.\]
  
  Since \(f_k\) is measurable, by Thm 4.1, we have \(\left\{ |f_k(x)| \leq m\right\}\) measurable for all \(k\), thus, by Thm 3.18, \( E_m = \bigcap_{k=1}^{\infty} \left\{ |f_k(x)| \leq m\right\}\).


  For the measurable \(E_M^c\), given \(\epsilon>0\), there exists an open set \(G\) s.t. 

  \[E_M^c \subset G \text{ and } |G \setminus E_M^c| < \epsilon/2.\]

  Let \(F = G^c\) be a close set, we have

  \[F \subset E_M \text{ and }  |E_M  \setminus  F| = |E_M  \setminus  G^c| < \epsilon/2.\]


  Then, 

  \[|E \setminus F| = |E \setminus E_M| + |E_M \setminus F| < \epsilon\]

  and $|f_k(x)| \leq M$ for all $k$ and all $x \in F$ as required.  


  
  \begin{mybox}{}
    
    \setcounter{section}{3}
    \setcounter{theorem}{25}
      \begin{theorem}
        Let $\{E_k\}_{k=1}^{\infty}$ be a sequence of measurable sets.
        \begin{enumerate}[label=(\roman*)]
          \item If $E_k \subset E$, then $\lim_{k\to\infty} |E_k| = |E|$.
          \item If $E_k \supset E$ and $|E_k| < +\infty$ for some $k$, then $\lim_{k\to\infty} |E_k| = |E|$.
        \end{enumerate}
        \end{theorem}
      \end{mybox}

\end{document}