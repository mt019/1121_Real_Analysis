\documentclass[UTF8,a4paper,10pt]{article}

% \begin{equation*}
%   \begin{aligned}
%   \end{aligned}
% \end{equation*}

% \begin{mybox}{}
% \end{mybox}


% \begin{Problem}[]{}
% \end{Problem}

% \begin{solution}\,
% \end{solution}
  

% \begin{enumerate}[label=(\alph*)]
% \end{enumerate} 

% \setcounter{section}{3} 
% \setcounter{theorem}{3}

% \begin{theorem}\label{thm:3.4}
%   If $E = \bigcup_{k} E_k$ is a countable union of sets, then $|E|_e \leq \sum_{k} |E_k|_e$.
%   \end{theorem}

%  \footcite[][42]{Wheeden_Zygmund_2015}

\input{preamble.tex}

\begin{document}



  \begin{Problem}[]{Zygmund p59 exercise 05}
    Construct a subset of $[0, 1]$ in the same manner as the Cantor set, except that at the $k$-stage, each interval removed has length $\delta3^{-k}$, where $0 < \delta < 1$. Show that the set has measure $1 - \delta$.
    %  [Zygmund p48 exercise 5]


 
  \end{Problem}

  
\begin{solution}\,

  Construct a subset of \([0, 1]\) in the same manner as the Cantor set, except that at the \(k\)-th stage, each interval removed has length \(\delta 3^{-k}\), \(0 < \delta < 1\). Let \(F_k\) denote the union of the intervals left at the \(k\)-th stage.
  Now show that the resulting set (Fat Cantor Set) \(F = \bigcap_{k=1}^{\infty} F_k \) has positive measure \(1 - \delta\), and contains no intervals.
  
  By construction,
  \begin{equation*}
    \begin{aligned}
      |F_k| = 1 - \sum_{i=1}^{k} 2^{i-1}\delta(\frac{1}{3})^i.
    \end{aligned}
  \end{equation*}
  Since
  \begin{equation*}
    \begin{aligned}
      0\leq|F|_e\leq|F_k|_e,
    \end{aligned}
  \end{equation*}
  let \(k\to \infty\), we have 
  \begin{equation*}
    \begin{aligned}
      |F|_e = \lim_{k \to \infty}|F_k|_e = 1-\delta  .
    \end{aligned}
  \end{equation*}
  
  Since \(F\) cannot contain an interval of length greater than \(1/2^k\) for all \(k\), so \(F\) contains no intervals. 
  


\end{solution}

  
\begin{Problem}[]{Zygmund p60 exercise 25}

  Construct a measurable subset in $[0, 1]$ such that for every interval in $[0, 1]$, both $E \cap I$ and $E^c \cap I$ have some property. 
  % [Zygmund p49 exercise 25]
\end{Problem}


\begin{solution}\,

Therefore, we have
\begin{equation*}
  \begin{aligned}
    |I\cap F^c| > 0,  \forall I\subseteq[0,1] .
  \end{aligned}
\end{equation*}
However, by construction,
\begin{equation*}
  \begin{aligned}
    \exists I\subseteq[0,1]\text{ s.t. }|I\cap F^c| > 0.
  \end{aligned}
\end{equation*}

Construct another such set on each subinterval of the complement of F, and get the resulting set \(E\). Thus, \(E^c\) contains no intervals by construction. Therefore, \(\forall I\subseteq[0,1]\), we have
\begin{equation*}
  \begin{aligned}
    |I\cap E| > 0,\\
    |I\cap E^c| > 0.
  \end{aligned}
\end{equation*}

\end{solution}


\begin{Problem}[]{}
  Motivated by (3.7), define the inner measure of $E$ by $|E|_i = \sup |F|$, where
  the supremum is taken over all closed subsets $F$ of $E$. Show that
  \begin{enumerate}
    \item $|E|_i \leq |E|_e$,
    \item if $|E|_e < +\infty$, then $E$ is measurable if and only if $|E|_i = |E|_e$.
  \end{enumerate}
  (Use Lemma 3.22.)

\end{Problem}


\begin{mybox}{}

  \setcounter{section}{3} 
  \setcounter{theorem}{5}
  
  \begin{theorem}\label{thm:3.6}
    Let $E \subset \mathbb{R}^n$. Then, given $\varepsilon > 0$, there exists an open set $G$ such that
    $E \subset G$ and $|G|_e \leq |E|_e + \varepsilon$. Hence,
    % \[  \]

    \begin{equation}
      \tag{3.7}
      \begin{aligned}
        |E|_e = \inf |G|_e,
      \end{aligned}
    \end{equation}
    where the infimum is taken over all open sets $G$ containing $E$.
    
    \end{theorem}
    
    \begin{proof}
    We may assume that $|E_k|_e < +\infty$ for each $k = 1, 2, \ldots$, since otherwise, the conclusion is obvious. Fix $\varepsilon > 0$. Given $k$, choose intervals $I^{(k)}_j$ such that $E_k \subset \bigcup_j I^{(k)}_j$ and $\sum_j v(I^{(k)}_j) < |E_k|_e + \varepsilon 2^{-k}$.
    
    Since $E \subset \bigcup_{j,k} I^{(k)}_j$, we have $|E|_e \leq \sum_{j,k} v(I^{(k)}_j) = \sum_k \sum_j v(I^{(k)}_j)$. Therefore,
    \[|E|_e \leq \sum_k (|E_k|_e + \varepsilon 2^{-k}) = \sum_k |E_k|_e + \varepsilon,\]
    and the result follows by letting $\varepsilon \to 0$.\footcite[][42]{Wheeden_Zygmund_2015}
    \end{proof}
     

\end{mybox}


\begin{Problem}[]{}
  Construct a continuous function $f$ such that $f$ on $[0, 1]$ is not of bounded variation on any interval. (Hint: Modify the Cantor-Lebesgue function) [Zygmund p49 exercise 26]

\end{Problem}

\begin{Problem}[]{}
  Show that there are disjoint sets $E_i \subset \mathbb{R}$, where $i = 1, 2, \ldots$, such that $\left|\bigcup_{i=1}^{\infty} E_i\right|_e < \sum_{i=1}^{\infty}|E_i|_e.$ [Zygmund p48 exercise 20]

\end{Problem}

  \begin{solution}\,



  \end{solution}
  \pagebreak





\end{document}