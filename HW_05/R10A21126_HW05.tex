\documentclass[UTF8,a4paper,10pt]{article}

% \begin{equation*}
%   \begin{aligned}
%   \end{aligned}
% \end{equation*}

% \begin{mybox}{}
% \end{mybox}


% \begin{Problem}[]{}
% \end{Problem}

% \begin{solution}\,
% \end{solution}
  

% \begin{enumerate}[label=(\alph*)]
% \end{enumerate} 

% \setcounter{section}{3} 
% \setcounter{theorem}{3}

% \begin{theorem}\label{thm:3.4}
%   If $E = \bigcup_{k} E_k$ is a countable union of sets, then $|E|_e \leq \sum_{k} |E_k|_e$.
%   \end{theorem}

%  \footcite[][42]{Wheeden_Zygmund_2015}

% \documentclass[UTF8,a4paper,14pt]{article}
% \usepackage[utf8]{inputenc}
\usepackage{amsmath}
% \usepackage{algorithm,algorithmic}
\usepackage[linesnumbered,ruled,vlined]{algorithm2e}

% \usepackage{algorithmicx}
% \usepackage{algpseudocode}
\usepackage{hyperref}

% \usepackage{algpseudocode}
\usepackage{amssymb}
\usepackage{amsfonts}
%for 字體
%https://tug.org/FontCatalogue/
% \usepackage[T1]{fontenc}
% \usepackage{tgbonum}
% \usepackage[bitstream-charter]{mathdesign}
% \usepackage[T1]{fontenc}
% \usepackage{bm}#粗體
%\usepackage{boondox-calo}
\usepackage{textcomp}
\usepackage{fancyhdr}%导入fancyhdrf包
\usepackage{ctex}%导入ctex包
\usepackage{enumitem} %for在Latex使用條列式清單
\usepackage{varwidth}
\usepackage{soul} %for \ul
\usepackage{comment}%\begin{comment}\end{comment}
\usepackage{cancel}%\cancel{}
%\usepackage{unicode-math}

\usepackage[dvipsnames, svgnames, x_11names]{xcolor}

\usepackage[low-sup]{subdepth}
\usepackage{subdepth}

\newcommand{\indep}{\Perp \!\!\! \Perp}

\usepackage{amsthm}
\DeclareMathOperator{\E}{\mathbb{E}}
\DeclareMathOperator{\Var}{\textbf{Var}}
\DeclareMathOperator{\Cov}{\textbf{Cov}}
\DeclareMathOperator{\Cor}{\textbf{Cor}}
\DeclareMathOperator{\X}{\mathbf{X}}
\DeclareMathOperator{\Pro}{\mathbf{P}}
\DeclareMathOperator{\M}{\mathbf{M}}
\DeclareMathOperator{\Id}{\mathbf{I}}
\DeclareMathOperator{\Y}{\mathbf{Y}}
\DeclareMathOperator{\MSFE}{\mathbf{MSFE}}
\DeclareMathOperator{\e}{\mathbb{e}}
\DeclareMathOperator{\V}{\mathbf{V}} 
\DeclareMathOperator{\tr}{\text{tr}}
\DeclareMathOperator{\A}{\textbf{A}}
\DeclareMathOperator{\diag}{diag}
\DeclareRobustCommand{\rchi}{{\mathpalette\irchi\relax}}
\newcommand{\irchi}[2]{\raisebox{\depth}{$#1\chi$}} % inner command, used by \rchi
\DeclareMathOperator*{\argmax}{arg\,max}
\DeclareMathOperator*{\argmin}{arg\,min}

\DeclareMathSizes{20}{10}{10}{5}

\usepackage[a4paper, margin=1in]{geometry}
% \setlength\parskip{5ex}% it would be better define distance in ex (5ex) 
                         %  or in pt, pc, mm, etc (see edit below)

\setlength{\parindent}{0pt}
\usepackage{array, makecell} %


%中英文設定
%\usepackage{fontspec}
% \setmainfont{TeX Gyre Termes}
% \usepackage{xeCJK} %引用中文字的指令集
% %\setCJKmainfont{PMingLiU}
% \setCJKmainfont{DFKai-SB}





% \setmainfont{Times New Roman}
% \setCJKmonofont{DFKai-SB}
\pagenumbering{arabic}%设置页码格式
\pagestyle{fancy}
\fancyhead{} % 初始化页眉
\usepackage{advdate}

% \newcommand{\yesterday}{{\AdvanceDate[-1]\today}}

\fancyhead[C]{Real Analysis\quad HW 03\quad  R10A21126\quad  WANG YIFAN\quad   \today}
%\fancyhead[LE]{\textsl{\rightmark}}
%\fancyfoot{} % 初始化页脚
%\fancyfoot[LO]{奇数页左页脚}
%\fancyfoot[LE]{偶数页左页脚}
%\fancyfoot[RO]{奇数页右页脚}
%\fancyfoot[RE]{偶数页右页脚}

% \title{{Econometrics HW 05}}
% \author{R10A21126}
% \date{\today}

%\fancyhf{}
\usepackage{lastpage}
\cfoot{Page \thepage \hspace{1pt} of\, \pageref{LastPage}}

\renewcommand{\headrulewidth}{0.1pt}%分隔线宽度4磅
%\renewcommand{\footrulewidth}{4pt}

\allowdisplaybreaks
\usepackage[english]{babel}
%\usepackage{amsthm}
\newtheorem{theorem}{Theorem}[section]
\newtheorem{corollary}{Corollary}[theorem]
\newtheorem{lemma}[theorem]{Lemma}


\usepackage[most]{tcolorbox}

\definecolor{babyblue}{rgb}{0.54, 0.81, 0.94}

\newtcolorbox[auto counter]{mybox}[1]{
  % Define a new tcolorbox style with custom paragraph spacing
  before upper={\parskip=10pt},
    after upper={\parskip=10pt},
    enhanced,
    arc= 1 mm,boxrule=1.5pt,
    colframe=babyblue!80!pink,
    colback=white,
    coltitle=black,
    % colback=blue!5!white,
    attach boxed title to top left=
    {xshift=1.5em,yshift=-\tcboxedtitleheight/2},
    boxed title style={size=small,
    % frame hidden,
    colback=White},
    top=0.15in,
    % fonttitle=\bfseries,
    title= {#1},
    breakable
  }

\newtcolorbox[auto counter]{Problem}[2][]{
    enhanced,drop shadow={Pink!50!white},
    colframe=pink!80!white,
    fonttitle=\bfseries,
    title=Problem ~\thetcbcounter. #2,
    %separator sign={.},
    coltitle=black,
    colback=pink!15,
    top=0.15in,
    breakable
  }

\newenvironment{solution}
  {\renewcommand\qedsymbol{$\blacksquare$}\begin{proof}[Solution]}
  {\end{proof}}

\theoremstyle{definition}
\newtheorem{definition}{Definition}[section]

%\theoremstyle{notation}
\newtheorem*{notation}{\underline{Notation}}
%\newtheorem*{convention}{\underline{Convention}}
\newtheorem*{convention}{\underline{Convention}}

\theoremstyle{remark}
\newtheorem*{remark}{Remark}

\newenvironment{amatrix}[2]{%% [2] for 2 parameters 
  \left[\begin{array}
    %{cc\,|\,cc}
    %  {@{}*{#2}{c}\,|\,c*{#1}{c}}
     {{}*{#1}{c}\,|\,c*{#2}{c}}
}{%
  \end{array}\right]
}
% For augmented matrix  
%https://tex.stackexchange.com/questions/2233/whats-the-best-way-make-an-augmented-coefficient-matrix


% defines the paragraph spacing
\setlength{\parskip}{0.5em}


\usepackage[sorting=none, citestyle=verbose-inote,backref=true,ibidtracker=context,mincrossrefs=99,backend=biber, 
url = false,
doi = false, isbn=false,]{biblatex}

\addbibresource{R10A21126.bib}

\usepackage{graphicx}
\graphicspath{ {images/} }
\usepackage{caption}

% global change
\SetKwInput{KwData}{Input}
\SetKwInput{KwResult}{Output}
% https://tex.stackexchange.com/questions/299771/how-do-i-rename-data-from-kwdata-and-result-from-kwresult-in-begi

\hypersetup{hidelinks}

\begin{document}



  \begin{Problem}[]{Zygmund p59 exercise 05}
    Construct a subset of $[0, 1]$ in the same manner as the Cantor set, except that at the $k$-stage, each interval removed has length $\delta3^{-k}$, where $0 < \delta < 1$. Show that the set has measure $1 - \delta$.
    %  [Zygmund p48 exercise 5]


 
  \end{Problem}

  
\begin{solution}\,

  Construct a subset of \([0, 1]\) in the same manner as the Cantor set, except that at the \(k\)-th stage, each interval removed has length \(\delta 3^{-k}\), \(0 < \delta < 1\). Let \(F_k\) denote the union of the intervals left at the \(k\)-th stage.
  Now show that the resulting set (Fat Cantor Set) \(F = \bigcap_{k=1}^{\infty} F_k \) has positive measure \(1 - \delta\), and contains no intervals.
  
  By construction,
  \begin{equation*}
    \begin{aligned}
      |F_k| = 1 - \sum_{i=1}^{k} 2^{i-1}\delta(\frac{1}{3})^i.
    \end{aligned}
  \end{equation*}
  Since
  \begin{equation*}
    \begin{aligned}
      0\leq|F|_e\leq|F_k|_e,
    \end{aligned}
  \end{equation*}
  let \(k\to \infty\), we have 
  \begin{equation*}
    \begin{aligned}
      |F|_e = \lim_{k \to \infty}|F_k|_e = 1-\delta  .
    \end{aligned}
  \end{equation*}
  
  Since \(F\) cannot contain an interval of length greater than \(1/2^k\) for all \(k\), so \(F\) contains no intervals. 
  


\end{solution}

  
\begin{Problem}[]{Zygmund p60 exercise 25}

  Construct a measurable subset in $[0, 1]$ such that for every interval in $[0, 1]$, both $E \cap I$ and $E^c \cap I$ have some property. 
  % [Zygmund p49 exercise 25]
\end{Problem}


\begin{solution}\,

Therefore, we have
\begin{equation*}
  \begin{aligned}
    |I\cap F^c| > 0,  \forall I\subseteq[0,1] .
  \end{aligned}
\end{equation*}
However, by construction,
\begin{equation*}
  \begin{aligned}
    \exists I\subseteq[0,1]\text{ s.t. }|I\cap F^c| > 0.
  \end{aligned}
\end{equation*}

Construct another such set on each subinterval of the complement of F, and get the resulting set \(E\). Thus, \(E^c\) contains no intervals by construction. Therefore, \(\forall I\subseteq[0,1]\), we have
\begin{equation*}
  \begin{aligned}
    |I\cap E| > 0,\\
    |I\cap E^c| > 0.
  \end{aligned}
\end{equation*}

\end{solution}


\begin{Problem}[]{}
  Motivated by (3.7), define the inner measure of $E$ by $|E|_i = \sup |F|$, where
  the supremum is taken over all closed subsets $F$ of $E$. Show that
  \begin{enumerate}
    \item $|E|_i \leq |E|_e$,
    \item if $|E|_e < +\infty$, then $E$ is measurable if and only if $|E|_i = |E|_e$.
  \end{enumerate}
  (Use Lemma 3.22.)

\end{Problem}


\begin{mybox}{}

  \setcounter{section}{3} 
  \setcounter{theorem}{5}
  
  \begin{theorem}\label{thm:3.6}
    Let $E \subset \mathbb{R}^n$. Then, given $\varepsilon > 0$, there exists an open set $G$ such that
    $E \subset G$ and $|G|_e \leq |E|_e + \varepsilon$. Hence,
    % \[  \]

    \begin{equation}
      \tag{3.7}
      \begin{aligned}
        |E|_e = \inf |G|_e,
      \end{aligned}
    \end{equation}
    where the infimum is taken over all open sets $G$ containing $E$.
    
    \end{theorem}
    
    \begin{proof}
    We may assume that $|E_k|_e < +\infty$ for each $k = 1, 2, \ldots$, since otherwise, the conclusion is obvious. Fix $\varepsilon > 0$. Given $k$, choose intervals $I^{(k)}_j$ such that $E_k \subset \bigcup_j I^{(k)}_j$ and $\sum_j v(I^{(k)}_j) < |E_k|_e + \varepsilon 2^{-k}$.
    
    Since $E \subset \bigcup_{j,k} I^{(k)}_j$, we have $|E|_e \leq \sum_{j,k} v(I^{(k)}_j) = \sum_k \sum_j v(I^{(k)}_j)$. Therefore,
    \[|E|_e \leq \sum_k (|E_k|_e + \varepsilon 2^{-k}) = \sum_k |E_k|_e + \varepsilon,\]
    and the result follows by letting $\varepsilon \to 0$.\footcite[][42]{Wheeden_Zygmund_2015}
    \end{proof}
     

\end{mybox}


\begin{Problem}[]{}
  Construct a continuous function $f$ such that $f$ on $[0, 1]$ is not of bounded variation on any interval. (Hint: Modify the Cantor-Lebesgue function) [Zygmund p49 exercise 26]

\end{Problem}

\begin{Problem}[]{}
  Show that there are disjoint sets $E_i \subset \mathbb{R}$, where $i = 1, 2, \ldots$, such that $\left|\bigcup_{i=1}^{\infty} E_i\right|_e < \sum_{i=1}^{\infty}|E_i|_e.$ [Zygmund p48 exercise 20]

\end{Problem}

  \begin{solution}\,



  \end{solution}
  \pagebreak





\end{document}