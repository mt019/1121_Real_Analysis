\documentclass[UTF8,a4paper,10pt]{article}

% \begin{equation*}
%   \begin{aligned}
%   \end{aligned}
% \end{equation*}

% \begin{mybox}{}
% \end{mybox}


% \begin{Problem}[]{}
% \end{Problem}

% \begin{solution}\,
% \end{solution}
  

% \begin{enumerate}[label=(\alph*)]
% \end{enumerate} 

% \setcounter{section}{3} 
% \setcounter{theorem}{3}

% \begin{theorem}\label{thm:3.4}
%   If $E = \bigcup_{k} E_k$ is a countable union of sets, then $|E|_e \leq \sum_{k} |E_k|_e$.
%   \end{theorem}

%  \footcite[][42]{Wheeden_Zygmund_2015}

% \documentclass[UTF8,a4paper,14pt]{article}
% \usepackage[utf8]{inputenc}
\usepackage{amsmath}
% \usepackage{algorithm,algorithmic}
\usepackage[linesnumbered,ruled,vlined]{algorithm2e}

% \usepackage{algorithmicx}
% \usepackage{algpseudocode}
\usepackage{hyperref}

% \usepackage{algpseudocode}
\usepackage{amssymb}
\usepackage{amsfonts}
%for 字體
%https://tug.org/FontCatalogue/
% \usepackage[T1]{fontenc}
% \usepackage{tgbonum}
% \usepackage[bitstream-charter]{mathdesign}
% \usepackage[T1]{fontenc}
% \usepackage{bm}#粗體
%\usepackage{boondox-calo}
\usepackage{textcomp}
\usepackage{fancyhdr}%导入fancyhdrf包
\usepackage{ctex}%导入ctex包
\usepackage{enumitem} %for在Latex使用條列式清單
\usepackage{varwidth}
\usepackage{soul} %for \ul
\usepackage{comment}%\begin{comment}\end{comment}
\usepackage{cancel}%\cancel{}
%\usepackage{unicode-math}

\usepackage[dvipsnames, svgnames, x_11names]{xcolor}

\usepackage[low-sup]{subdepth}
\usepackage{subdepth}

\newcommand{\indep}{\Perp \!\!\! \Perp}

\usepackage{amsthm}
\DeclareMathOperator{\E}{\mathbb{E}}
\DeclareMathOperator{\Var}{\textbf{Var}}
\DeclareMathOperator{\Cov}{\textbf{Cov}}
\DeclareMathOperator{\Cor}{\textbf{Cor}}
\DeclareMathOperator{\X}{\mathbf{X}}
\DeclareMathOperator{\Pro}{\mathbf{P}}
\DeclareMathOperator{\M}{\mathbf{M}}
\DeclareMathOperator{\Id}{\mathbf{I}}
\DeclareMathOperator{\Y}{\mathbf{Y}}
\DeclareMathOperator{\MSFE}{\mathbf{MSFE}}
\DeclareMathOperator{\e}{\mathbb{e}}
\DeclareMathOperator{\V}{\mathbf{V}} 
\DeclareMathOperator{\tr}{\text{tr}}
\DeclareMathOperator{\A}{\textbf{A}}
\DeclareMathOperator{\diag}{diag}
\DeclareRobustCommand{\rchi}{{\mathpalette\irchi\relax}}
\newcommand{\irchi}[2]{\raisebox{\depth}{$#1\chi$}} % inner command, used by \rchi
\DeclareMathOperator*{\argmax}{arg\,max}
\DeclareMathOperator*{\argmin}{arg\,min}

\DeclareMathSizes{20}{10}{10}{5}

\usepackage[a4paper, margin=1in]{geometry}
% \setlength\parskip{5ex}% it would be better define distance in ex (5ex) 
                         %  or in pt, pc, mm, etc (see edit below)

\setlength{\parindent}{0pt}
\usepackage{array, makecell} %


%中英文設定
%\usepackage{fontspec}
% \setmainfont{TeX Gyre Termes}
% \usepackage{xeCJK} %引用中文字的指令集
% %\setCJKmainfont{PMingLiU}
% \setCJKmainfont{DFKai-SB}





% \setmainfont{Times New Roman}
% \setCJKmonofont{DFKai-SB}
\pagenumbering{arabic}%设置页码格式
\pagestyle{fancy}
\fancyhead{} % 初始化页眉
\usepackage{advdate}

% \newcommand{\yesterday}{{\AdvanceDate[-1]\today}}

\fancyhead[C]{Real Analysis\quad HW 03\quad  R10A21126\quad  WANG YIFAN\quad   \today}
%\fancyhead[LE]{\textsl{\rightmark}}
%\fancyfoot{} % 初始化页脚
%\fancyfoot[LO]{奇数页左页脚}
%\fancyfoot[LE]{偶数页左页脚}
%\fancyfoot[RO]{奇数页右页脚}
%\fancyfoot[RE]{偶数页右页脚}

% \title{{Econometrics HW 05}}
% \author{R10A21126}
% \date{\today}

%\fancyhf{}
\usepackage{lastpage}
\cfoot{Page \thepage \hspace{1pt} of\, \pageref{LastPage}}

\renewcommand{\headrulewidth}{0.1pt}%分隔线宽度4磅
%\renewcommand{\footrulewidth}{4pt}

\allowdisplaybreaks
\usepackage[english]{babel}
%\usepackage{amsthm}
\newtheorem{theorem}{Theorem}[section]
\newtheorem{corollary}{Corollary}[theorem]
\newtheorem{lemma}[theorem]{Lemma}


\usepackage[most]{tcolorbox}

\definecolor{babyblue}{rgb}{0.54, 0.81, 0.94}

\newtcolorbox[auto counter]{mybox}[1]{
  % Define a new tcolorbox style with custom paragraph spacing
  before upper={\parskip=10pt},
    after upper={\parskip=10pt},
    enhanced,
    arc= 1 mm,boxrule=1.5pt,
    colframe=babyblue!80!pink,
    colback=white,
    coltitle=black,
    % colback=blue!5!white,
    attach boxed title to top left=
    {xshift=1.5em,yshift=-\tcboxedtitleheight/2},
    boxed title style={size=small,
    % frame hidden,
    colback=White},
    top=0.15in,
    % fonttitle=\bfseries,
    title= {#1},
    breakable
  }

\newtcolorbox[auto counter]{Problem}[2][]{
    enhanced,drop shadow={Pink!50!white},
    colframe=pink!80!white,
    fonttitle=\bfseries,
    title=Problem ~\thetcbcounter. #2,
    %separator sign={.},
    coltitle=black,
    colback=pink!15,
    top=0.15in,
    breakable
  }

\newenvironment{solution}
  {\renewcommand\qedsymbol{$\blacksquare$}\begin{proof}[Solution]}
  {\end{proof}}

\theoremstyle{definition}
\newtheorem{definition}{Definition}[section]

%\theoremstyle{notation}
\newtheorem*{notation}{\underline{Notation}}
%\newtheorem*{convention}{\underline{Convention}}
\newtheorem*{convention}{\underline{Convention}}

\theoremstyle{remark}
\newtheorem*{remark}{Remark}

\newenvironment{amatrix}[2]{%% [2] for 2 parameters 
  \left[\begin{array}
    %{cc\,|\,cc}
    %  {@{}*{#2}{c}\,|\,c*{#1}{c}}
     {{}*{#1}{c}\,|\,c*{#2}{c}}
}{%
  \end{array}\right]
}
% For augmented matrix  
%https://tex.stackexchange.com/questions/2233/whats-the-best-way-make-an-augmented-coefficient-matrix


% defines the paragraph spacing
\setlength{\parskip}{0.5em}


\usepackage[sorting=none, citestyle=verbose-inote,backref=true,ibidtracker=context,mincrossrefs=99,backend=biber, 
url = false,
doi = false, isbn=false,]{biblatex}

\addbibresource{R10A21126.bib}

\usepackage{graphicx}
\graphicspath{ {images/} }
\usepackage{caption}

% global change
\SetKwInput{KwData}{Input}
\SetKwInput{KwResult}{Output}
% https://tex.stackexchange.com/questions/299771/how-do-i-rename-data-from-kwdata-and-result-from-kwresult-in-begi

\hypersetup{hidelinks}

\begin{document}



  \begin{Problem}[]{Zygmund p59 exercise 05}
    Construct a subset of $[0, 1]$ in the same manner as the Cantor set, except that at the $k$-stage, each interval removed has length $\delta3^{-k}$, where $0 < \delta < 1$. Show that the set has measure $1 - \delta$.
    %  [Zygmund p48 exercise 5]


 
  \end{Problem}

  
\begin{solution}\,

  Construct a subset of \([0, 1]\) in the same manner as the Cantor set, except that at the \(k\)-th stage, each interval removed has length \(\delta 3^{-k}\), \(0 < \delta < 1\). Let \(F_k\) denote the union of the intervals left at the \(k\)-th stage.
  Now show that the resulting set (Fat Cantor Set) \(F = \bigcap_{k=1}^{\infty} F_k \) has positive measure \(1 - \delta\), and contains no intervals.
  
  By construction,
  \begin{equation*}
    \begin{aligned}
      |F_k| = 1 - \sum_{i=1}^{k} 2^{i-1}\delta(\frac{1}{3})^i.
    \end{aligned}
  \end{equation*}
  Since
  \begin{equation*}
    \begin{aligned}
      0\leq|F|_e\leq|F_k|_e,
    \end{aligned}
  \end{equation*}
  let \(k\to \infty\), we have 
  \begin{equation*}
    \begin{aligned}
      |F|_e = \lim_{k \to \infty}|F_k|_e = 1-\delta  .
    \end{aligned}
  \end{equation*}
  
  Since \(F\) cannot contain an interval of length greater than \(1/2^k\) for all \(k\), so \(F\) contains no intervals. 
  


\end{solution}

  
\begin{Problem}[]{Zygmund p60 exercise 25}

  Construct a measurable subset \(E\) of \([0, 1]\) such that for every subinterval \(I\), both \(E \cap  I\) and \(I - E\) have positive measure. (Take a Cantor-type subset of \([0, 1]\) with positive measure [see Exercise 5], and on each subinterval of the complement of this set, construct another such set, and so on. The measures can be arranged so that the union of all the sets has the desired property.) See also Exercise 21 of Chapter 4.

\end{Problem}


\begin{solution}\,

Therefore, we have
\begin{equation*}
  \begin{aligned}
    |I\cap F^c| > 0,  \forall I\subseteq[0,1] .
  \end{aligned}
\end{equation*}
However, by construction,
\begin{equation*}
  \begin{aligned}
    \exists I\subseteq[0,1]\text{ s.t. }|I\cap F| = 0.
  \end{aligned}
\end{equation*}

Construct another such set on each subinterval of the complement of F, and get the resulting set \(E\). Thus, \(E^c\) contains no intervals by construction. Therefore, \(\forall I\subseteq[0,1]\), we have
\begin{equation*}
  \begin{aligned}
    |I\cap E| > 0,\\
    |I\cap E^c| > 0.
  \end{aligned}
\end{equation*}

\end{solution}


\pagebreak
% Q3
\begin{Problem}[]{Zygmund p59 exercise 13}
  Motivated by (3.7), define the inner measure of $E$ by $|E|_i = \sup |F|$, where
  the supremum is taken over all closed subsets $F$ of $E$. Show that
  \begin{enumerate}
    \item $|E|_i \leq |E|_e$,
    \item if $|E|_e < +\infty$, then $E$ is measurable if and only if $|E|_i = |E|_e$.
  \end{enumerate}
  (Use Lemma 3.22.)

\end{Problem}


\begin{mybox}{}

  \setcounter{section}{3} 
  \setcounter{theorem}{5}
  
  \begin{theorem}\label{thm:3.6}
    Let $E \subset \mathbb{R}^n$. Then, given $\varepsilon > 0$, there exists an open set $G$ such that
    $E \subset G$ and $|G|_e \leq |E|_e + \varepsilon$. Hence,
    % \[  \]

    \begin{equation}
      \tag{3.7}
      \begin{aligned}
        |E|_e = \inf |G|_e,
      \end{aligned}
    \end{equation}
    where the infimum is taken over all open sets $G$ containing $E$.
    
    \end{theorem}
    
    \begin{proof}
    We may assume that $|E_k|_e < +\infty$ for each $k = 1, 2, \ldots$, since otherwise, the conclusion is obvious. Fix $\varepsilon > 0$. Given $k$, choose intervals $I^{(k)}_j$ such that $E_k \subset \bigcup_j I^{(k)}_j$ and $\sum_j v(I^{(k)}_j) < |E_k|_e + \varepsilon 2^{-k}$.
    
    Since $E \subset \bigcup_{j,k} I^{(k)}_j$, we have $|E|_e \leq \sum_{j,k} v(I^{(k)}_j) = \sum_k \sum_j v(I^{(k)}_j)$. Therefore,
    \[|E|_e \leq \sum_k (|E_k|_e + \varepsilon 2^{-k}) = \sum_k |E_k|_e + \varepsilon,\]
    and the result follows by letting $\varepsilon \to 0$.\footcite[][42]{Wheeden_Zygmund_2015}
    \end{proof}
     

\end{mybox}

\pagebreak

% Q4

\begin{Problem}[]{Zygmund p60 exercise 26}  

  Construct a continuous function \(f\) on \([0, 1]\), which is not of bounded variation on any subinterval. 

\end{Problem}


\begin{solution}\,


  The construction follows the pattern of the  Cantor–Lebesgue function with some modifications. At the first stage, make the corresponding function increase to 2/3 (rather than 1/2) in (0, 1/3), then make it decrease by 1/3 in (1/3, 2/3), and then increase again 2/3 in (2/3, 1). 
  
  In the next stage, for each subinterval, apply a similar pattern based on the behavior of the function in the preceding interval. If the function was increasing in the previous interval, make it decrease by a certain amount in the middle of the current interval. If it was decreasing, make it increase by a certain amount. 
  
  
  Repeat this process infinitely.
  It should be sufficient to ensure that the function's oscillations are unbounded.

  The construction at each stage should make sure that the function is continuous.
  With these alternating operations, the key to the unbounded variation is the fact that the function switches between increasing and decreasing behavior within each subinterval, and the oscillations do not converge to a limit.

\end{solution}

% Q5
% \pagebreak

\begin{Problem}[]{Zygmund p60 exercise 20}
  Show that there are disjoint sets $E_i \subset \mathbb{R}$, where $i = 1, 2, \ldots$, such that $\left|\bigcup_{i=1}^{\infty} E_i\right|_e < \sum_{i=1}^{\infty}|E_i|_e.$ 

\end{Problem}


\begin{mybox}{}
    \setcounter{theorem}{36}
    \begin{lemma}
      Let $E$ be a measurable subset of $\mathbb{R}^1$ with $|E| > 0$. Then the set of differences $\{d : d = x - y, x \in E, y \in E\}$ contains an interval centered at the origin.
      \end{lemma}
      

    % \setcounter{theorem}{37}
    \begin{theorem}[Vitali]
      There exist nonmeasurable sets.
      \end{theorem}
    
      \begin{proof}
        We define an equivalence relation on the real line by saying that $x$ and $y$ are equivalent if $x - y$ is rational. The equivalence classes then have the form
        \[
        E_x = \{x + r : r \text{ is rational}\}.
        \]
        Two classes $E_x$ and $E_y$ are either identical or disjoint;
        therefore, one equivalence class consists of all the rational numbers, and the
  other distinct classes are disjoint sets of irrational numbers. The number of
  distinct equivalence classes is uncountable since each class is countable, but
  the union of all the classes is uncountable (this union being the entire line).
  Using Zermelo's axiom, let $E$ be a set consisting of exactly one element
  from each distinct equivalence class. Since any two points of $E$ must differ by
  an irrational number, the set $\{d : d = x - y, x \in E, y \in E\}$ cannot contain an
  interval. By Lemma 3.37, it follows that either $E$ is not measurable or $|E| = 0$.
  Since the union of the translates of $E$ by every rational number is all of $\mathbb{R}^1$, $\mathbb{R}^1$
  would have measure zero if $E$ did. We conclude that $E$ is not measurable.

      \end{proof}

    \end{mybox}

\pagebreak
\begin{solution}\,

  Let $E$ be a set consisting of exactly one element
  from each distinct equivalence class in \([0,1]\).
  \begin{equation*}
    \begin{aligned}
      \forall x\neq y\in E,\\
       x-y \notin \mathbb{Q}.
    \end{aligned}
  \end{equation*}

  Let \(\{r_i\}_{i=1}^{\infty}\) be the enumeration of rational numbers in the interval \([0, 1]\).
  
  In other words, \(\{r_i\}_{i=1}^{\infty}=\mathbb{Q} \cap [0,1]\).
  
  Let

\begin{equation*}
  \begin{aligned}
    E_i = E+r_i.
  \end{aligned}
\end{equation*}

 \(E_i\) are disjoint.

 
\begin{mybox}{}
  \(\forall i \neq j \), \(E_i\cap E_j = \emptyset\).

  prove by contradiction:

  If \(\forall i \neq j \), \(E_i\cap E_j \neq \emptyset\),

  then \(\exists x\in E_i\cap E_j \),
  which means \( x\in E_i\) and \(x\in E_j\).

  \begin{equation*}
    \begin{aligned}
      &x\in E_i\Rightarrow x\in (E+r_i)\Rightarrow x-r_i\in E;\\
      &x\in E_j\Rightarrow x\in (E+r_j) \Rightarrow x-r_j\in E.
    \end{aligned}
  \end{equation*}

  \((x-r_i)-(x-r_j)\) is rational, contradictory to the construction of \(E\).

  Therefore, we can conclude that  \(\forall i \neq j \), \(E_i\cap E_j = \emptyset\).



\end{mybox}


\(E_i\) are nonmeasurable and \(\bigcup_{i=1}^{\infty} E_i \subseteq [0,2]\).

We have
\begin{equation*}
  \begin{aligned}
\left|\bigcup_{i=1}^{\infty} E_i\right|_e
\leq |[0,2]|_e = 2.
\end{aligned}
\end{equation*}

However, since \(|E_i|_e>0\), we observe that
\begin{equation*}
  \begin{aligned}
    \sum_{i=1}^{\infty}|E_i|_e = \infty.
\end{aligned}
\end{equation*}

Therefore, we can conclude that \(\left|\bigcup_{i=1}^{\infty} E_i\right|_e < \sum_{i=1}^{\infty}|E_i|_e\).


  \end{solution}

\end{document}