\documentclass[UTF8,a4paper,10pt]{article}

% \begin{equation*}
%   \begin{aligned}
%   \end{aligned}
% \end{equation*}

% \begin{mybox}{}
% \end{mybox}


% \begin{Problem}[]{}
% \end{Problem}

% \begin{solution}\,
% \end{solution}
  

% \begin{enumerate}[label=(\alph*)]
% \end{enumerate} 

% \setcounter{section}{3} 
% \setcounter{theorem}{3}

% \begin{theorem}\label{thm:3.4}
%   If $E = \bigsqcup_{k} E_k$ is a countable union of sets, then $|E|_e \leq \sum_{k} |E_k|_e$.
%   \end{theorem}

%  \footcite[][42]{Wheeden_Zygmund_2015}

% \documentclass[UTF8,a4paper,14pt]{article}
% \usepackage[utf8]{inputenc}
\usepackage{amsmath}
% \usepackage{algorithm,algorithmic}
\usepackage[linesnumbered,ruled,vlined]{algorithm2e}

% \usepackage{algorithmicx}
% \usepackage{algpseudocode}
\usepackage{hyperref}

% \usepackage{algpseudocode}
\usepackage{amssymb}
\usepackage{amsfonts}
%for 字體
%https://tug.org/FontCatalogue/
% \usepackage[T1]{fontenc}
% \usepackage{tgbonum}
% \usepackage[bitstream-charter]{mathdesign}
% \usepackage[T1]{fontenc}
% \usepackage{bm}#粗體
%\usepackage{boondox-calo}
\usepackage{textcomp}
\usepackage{fancyhdr}%导入fancyhdrf包
\usepackage{ctex}%导入ctex包
\usepackage{enumitem} %for在Latex使用條列式清單
\usepackage{varwidth}
\usepackage{soul} %for \ul
\usepackage{comment}%\begin{comment}\end{comment}
\usepackage{cancel}%\cancel{}
%\usepackage{unicode-math}

\usepackage[dvipsnames, svgnames, x_11names]{xcolor}

\usepackage[low-sup]{subdepth}
\usepackage{subdepth}

\newcommand{\indep}{\Perp \!\!\! \Perp}

\usepackage{amsthm}
\DeclareMathOperator{\E}{\mathbb{E}}
\DeclareMathOperator{\Var}{\textbf{Var}}
\DeclareMathOperator{\Cov}{\textbf{Cov}}
\DeclareMathOperator{\Cor}{\textbf{Cor}}
\DeclareMathOperator{\X}{\mathbf{X}}
\DeclareMathOperator{\Pro}{\mathbf{P}}
\DeclareMathOperator{\M}{\mathbf{M}}
\DeclareMathOperator{\Id}{\mathbf{I}}
\DeclareMathOperator{\Y}{\mathbf{Y}}
\DeclareMathOperator{\MSFE}{\mathbf{MSFE}}
\DeclareMathOperator{\e}{\mathbb{e}}
\DeclareMathOperator{\V}{\mathbf{V}} 
\DeclareMathOperator{\tr}{\text{tr}}
\DeclareMathOperator{\A}{\textbf{A}}
\DeclareMathOperator{\diag}{diag}
\DeclareRobustCommand{\rchi}{{\mathpalette\irchi\relax}}
\newcommand{\irchi}[2]{\raisebox{\depth}{$#1\chi$}} % inner command, used by \rchi
\DeclareMathOperator*{\argmax}{arg\,max}
\DeclareMathOperator*{\argmin}{arg\,min}

\DeclareMathSizes{20}{10}{10}{5}

\usepackage[a4paper, margin=1in]{geometry}
% \setlength\parskip{5ex}% it would be better define distance in ex (5ex) 
                         %  or in pt, pc, mm, etc (see edit below)

\setlength{\parindent}{0pt}
\usepackage{array, makecell} %


%中英文設定
%\usepackage{fontspec}
% \setmainfont{TeX Gyre Termes}
% \usepackage{xeCJK} %引用中文字的指令集
% %\setCJKmainfont{PMingLiU}
% \setCJKmainfont{DFKai-SB}





% \setmainfont{Times New Roman}
% \setCJKmonofont{DFKai-SB}
\pagenumbering{arabic}%设置页码格式
\pagestyle{fancy}
\fancyhead{} % 初始化页眉
\usepackage{advdate}

% \newcommand{\yesterday}{{\AdvanceDate[-1]\today}}

\fancyhead[C]{Real Analysis\quad HW 03\quad  R10A21126\quad  WANG YIFAN\quad   \today}
%\fancyhead[LE]{\textsl{\rightmark}}
%\fancyfoot{} % 初始化页脚
%\fancyfoot[LO]{奇数页左页脚}
%\fancyfoot[LE]{偶数页左页脚}
%\fancyfoot[RO]{奇数页右页脚}
%\fancyfoot[RE]{偶数页右页脚}

% \title{{Econometrics HW 05}}
% \author{R10A21126}
% \date{\today}

%\fancyhf{}
\usepackage{lastpage}
\cfoot{Page \thepage \hspace{1pt} of\, \pageref{LastPage}}

\renewcommand{\headrulewidth}{0.1pt}%分隔线宽度4磅
%\renewcommand{\footrulewidth}{4pt}

\allowdisplaybreaks
\usepackage[english]{babel}
%\usepackage{amsthm}
\newtheorem{theorem}{Theorem}[section]
\newtheorem{corollary}{Corollary}[theorem]
\newtheorem{lemma}[theorem]{Lemma}


\usepackage[most]{tcolorbox}

\definecolor{babyblue}{rgb}{0.54, 0.81, 0.94}

\newtcolorbox[auto counter]{mybox}[1]{
  % Define a new tcolorbox style with custom paragraph spacing
  before upper={\parskip=10pt},
    after upper={\parskip=10pt},
    enhanced,
    arc= 1 mm,boxrule=1.5pt,
    colframe=babyblue!80!pink,
    colback=white,
    coltitle=black,
    % colback=blue!5!white,
    attach boxed title to top left=
    {xshift=1.5em,yshift=-\tcboxedtitleheight/2},
    boxed title style={size=small,
    % frame hidden,
    colback=White},
    top=0.15in,
    % fonttitle=\bfseries,
    title= {#1},
    breakable
  }

\newtcolorbox[auto counter]{Problem}[2][]{
    enhanced,drop shadow={Pink!50!white},
    colframe=pink!80!white,
    fonttitle=\bfseries,
    title=Problem ~\thetcbcounter. #2,
    %separator sign={.},
    coltitle=black,
    colback=pink!15,
    top=0.15in,
    breakable
  }

\newenvironment{solution}
  {\renewcommand\qedsymbol{$\blacksquare$}\begin{proof}[Solution]}
  {\end{proof}}

\theoremstyle{definition}
\newtheorem{definition}{Definition}[section]

%\theoremstyle{notation}
\newtheorem*{notation}{\underline{Notation}}
%\newtheorem*{convention}{\underline{Convention}}
\newtheorem*{convention}{\underline{Convention}}

\theoremstyle{remark}
\newtheorem*{remark}{Remark}

\newenvironment{amatrix}[2]{%% [2] for 2 parameters 
  \left[\begin{array}
    %{cc\,|\,cc}
    %  {@{}*{#2}{c}\,|\,c*{#1}{c}}
     {{}*{#1}{c}\,|\,c*{#2}{c}}
}{%
  \end{array}\right]
}
% For augmented matrix  
%https://tex.stackexchange.com/questions/2233/whats-the-best-way-make-an-augmented-coefficient-matrix


% defines the paragraph spacing
\setlength{\parskip}{0.5em}


\usepackage[sorting=none, citestyle=verbose-inote,backref=true,ibidtracker=context,mincrossrefs=99,backend=biber, 
url = false,
doi = false, isbn=false,]{biblatex}

\addbibresource{R10A21126.bib}

\usepackage{graphicx}
\graphicspath{ {images/} }
\usepackage{caption}

% global change
\SetKwInput{KwData}{Input}
\SetKwInput{KwResult}{Output}
% https://tex.stackexchange.com/questions/299771/how-do-i-rename-data-from-kwdata-and-result-from-kwresult-in-begi

\hypersetup{hidelinks}

\begin{document}



  \begin{Problem}[]{Zygmund p76 exercise 05}
    Give an example to show that $\varphi(f(x))$ may not be measurable if $\varphi$ and $f$ are measurable and finite. (Let $F$ be the Cantor–Lebesgue function and let $f$ be its inverse, suitably defined. Let $\varphi$ be the characteristic function of a set of measure zero whose image under $F$ is not measurable.) Show that the same may be true even if $f$ is continuous. (Let $g(x) = x + F(x)$, where $F$ is the Cantor–Lebesgue function, and consider $f = g^{-1}$.) Cf. Exercise 22.

  \end{Problem}

 
\begin{Problem}[]{}

  Let \(\chi_{[0,1]} \) be the characteristic function of \([0,1]\). Show that there is no everywhere continuous function \(f\) on \(\R\) such that
\begin{equation*}
  \begin{aligned}
    f(x) = \chi_{[0,1]} (x) \,\,\text{almost everywhere.}
  \end{aligned}
\end{equation*}
\end{Problem}

\begin{solution}\,

  \begin{gather*}
    f(x) = \chi_{[0,1]}(x) \text{ almost everywhere.} \\
    \Updownarrow \\
    |\{x | f(x) \neq \chi_{[0,1]}(x)\}| = 0.
  \end{gather*}

\dotfill

  Suppose, for the sake of contradiction, that \(\exists f\) on \(\R\) s.t. 
  \begin{gather*}
    f(x) = \chi_{[0,1]}(x) \text{ almost everywhere.}
  \end{gather*}
  Without loss of generality, \(f(x) = 1, x\in [0,1]\).

  By the definition of continuous everywhere,
  \(\forall \epsilon>0, \exists \delta>0\), s.t. \(|x-0|<\delta \Rightarrow |f(x)-f(0)|<\epsilon\), which means \(|f(x)-1|<\epsilon\).

  \begin{align*}
    &\Rightarrow f(x) \neq 0 \text{ on } [-\delta, 0] \\
    &\Rightarrow f(x) \neq \chi_{[0,1]} \text{ on } [-\delta, 0], \delta>0.
\end{align*}
Thus, \(|\{x | f(x) \neq \chi_{[0,1]}(x)\}| \neq 0\), which contradicts the assumption that \(f(x) = \chi_{[0,1]}(x) \) a.e.

Therefore, we conclude that there is no everywhere continuous function \(f\) on \(\R\) such that \(f(x) = \chi_{[0,1]}(x) \) a.e.

\end{solution}

\pagebreak

\begin{Problem}[]{}
  Let \(\Gamma\subset \R^d \times\R\), \(\Gamma = \{(x,y)\in\R^d\times\R:y = f(x)\}\), and assume \(f\) is measurable on \(\R^d\). Show that \(\Gamma\) is a measurable subset of \(\R^{d+1}\), and \(|\Gamma| = 0.\)
\end{Problem}

\begin{solution}\,\\

  It suffices to prove that $|\Gamma|_e = 0$. Since $R^d$ is a countable union of almost disjoint cubes of side length $1$, it is enough to show that $|\Gamma'|_e = 0$, where 
  \[\Gamma' = \{(x, y) \in [0, 1]^d \times \mathbb{R} : y = f|_{[0,1]^d} (x)\}.\]

  Since we know $R = \bigsqcup _{k\in\mathbb{Z}}[k, k + 1)$, it follows that 
  \[\Gamma' = \bigsqcup_{k\in\mathbb{Z}}\{(x, y) \in [0, 1]^d \times [k, k + 1) : y = f|_{[0,1]^d} (x)\}.\]
  
  Again, it is sufficient to prove that $|\Gamma''|_e = 0$, where 
  \[\Gamma'' = \{(x, y) \in [0, 1]^d \times [0, 1) : y = f|_{[0,1]^d} (x)\}.\]
  
  For every $n \in \mathbb{N}$, we have $[0, 1) = \bigsqcup_{j=1}^n I_j$, where $I_j = \left[\frac{j - 1}{n}, \frac{j}{n}\right)$ for all $j \in \{1, 2, \ldots, n\}$.
  
  Since we know 
 \[\Gamma'' = \bigsqcup_{j=1}^n \left\{(x, y) \in [0, 1]^d \times I_j : y = f|_{[0,1]^d} (x)\right\},\]
  
  and $f|_{[0,1]^d}$ is measurable on $[0, 1]^d$, it follows that

  \begin{align*}
  |\Gamma''|_e &\le \sum_{j=1}^n \bigg|\left\{(x, y) \in [0, 1]^d \times I_j : y = f|_{[0,1]^d} (x)\right\}\bigg|_e \\
  &= \sum_{j=1}^n \bigg|f|_{[0,1]^d}^{\text{pre}} (I_j) \times I_j \bigg|_e \\
  &= \sum_{j=1}^n \bigg|f|_{[0,1]^d}^{\text{pre}} (I_j)\bigg| \cdot |I_j| \\
  &= \frac{1}{n} \sum_{j=1}^n \bigg|f|_{[0,1]^d}^{\text{pre}} (I_j)\bigg| \\
  &= \frac{1}{n} \bigg|\bigsqcup_{j=1}^n f|_{[0,1]^d}^{\text{pre}} (I_j)\bigg| \\
  &= \frac{1}{n} \bigg| f|_{[0,1]^d}^{\text{pre}} \left([0,1)\right)\bigg|  \le \frac{1}{n} \bigg|[0, 1]^d\bigg| = \frac{1}{n}
  \end{align*}

  for all $n \in \mathbb{N}$. Hence, we have

  \begin{equation*}
  |\Gamma''|_e \le \lim_{n \to \infty} \frac{1}{n} = 0.
  \end{equation*}

  Therefore, we obtain
  % \begin{equation*}
  \(|\Gamma''|_e = 0.\)
  % \end{equation*}
    
\end{solution}

\end{document}