\documentclass[UTF8,a4paper,10pt]{article}

% \begin{equation*}
%   \begin{aligned}
%   \end{aligned}
% \end{equation*}

% \begin{mybox}{}
% \end{mybox}


% \begin{Problem}[]{}
% \end{Problem}

% \begin{solution}\,
% \end{solution}
  

% \begin{enumerate}[label=(\alph*)]
% \end{enumerate} 

% \setcounter{section}{3} 
% \setcounter{theorem}{3}

% \begin{theorem}\label{thm:3.4}
%   If $E = \bigsqcup_{k} E_k$ is a countable union of sets, then $|E|_e \leq \sum_{k} |E_k|_e$.
%   \end{theorem}

%  \footcite[][42]{Wheeden_Zygmund_2015}

\input{preamble.tex}

\begin{document}



  \begin{Problem}[]{Zygmund p76 exercise 05}
    Give an example to show that $\varphi(f(x))$ may not be measurable if $\varphi$ and $f$ are measurable and finite. (Let $F$ be the Cantor–Lebesgue function and let $f$ be its inverse, suitably defined. Let $\varphi$ be the characteristic function of a set of measure zero whose image under $F$ is not measurable.) Show that the same may be true even if $f$ is continuous. (Let $g(x) = x + F(x)$, where $F$ is the Cantor–Lebesgue function, and consider $f = g^{-1}$.) Cf. Exercise 22.

  \end{Problem}

 
\begin{Problem}[]{}

  Let \(\chi_{[0,1]} \) be the characteristic function of \([0,1]\). Show that there is no everywhere continuous function \(f\) on \(\R\) such that
\begin{equation*}
  \begin{aligned}
    f(x) = \chi_{[0,1]} (x) \,\,\text{almost everywhere.}
  \end{aligned}
\end{equation*}
\end{Problem}

\begin{solution}\,

  \begin{gather*}
    f(x) = \chi_{[0,1]}(x) \text{ almost everywhere.} \\
    \Updownarrow \\
    |\{x | f(x) \neq \chi_{[0,1]}(x)\}| = 0.
  \end{gather*}

\dotfill

  Suppose, for the sake of contradiction, that \(\exists f\) on \(\R\) s.t. 
  \begin{gather*}
    f(x) = \chi_{[0,1]}(x) \text{ almost everywhere.}
  \end{gather*}
  Without loss of generality, \(f(x) = 1, x\in [0,1]\).

  By the definition of continuous everywhere,
  \(\forall \epsilon>0, \exists \delta>0\), s.t. \(|x-0|<\delta \Rightarrow |f(x)-f(0)|<\epsilon\), which means \(|f(x)-1|<\epsilon\).

  \begin{align*}
    &\Rightarrow f(x) \neq 0 \text{ on } [-\delta, 0] \\
    &\Rightarrow f(x) \neq \chi_{[0,1]} \text{ on } [-\delta, 0], \delta>0.
\end{align*}
Thus, \(|\{x | f(x) \neq \chi_{[0,1]}(x)\}| \neq 0\), which contradicts the assumption that \(f(x) = \chi_{[0,1]}(x) \) a.e.

Therefore, we conclude that there is no everywhere continuous function \(f\) on \(\R\) such that \(f(x) = \chi_{[0,1]}(x) \) a.e.

\end{solution}

\pagebreak

\begin{Problem}[]{}
  Let \(\Gamma\subset \R^d \times\R\), \(\Gamma = \{(x,y)\in\R^d\times\R:y = f(x)\}\), and assume \(f\) is measurable on \(\R^d\). Show that \(\Gamma\) is a measurable subset of \(\R^{d+1}\), and \(|\Gamma| = 0.\)
\end{Problem}

\begin{solution}\,\\

  It suffices to prove that $|\Gamma|_e = 0$. Since $R^d$ is a countable union of almost disjoint cubes of side length $1$, it is enough to show that $|\Gamma'|_e = 0$, where 
  \[\Gamma' = \{(x, y) \in [0, 1]^d \times \mathbb{R} : y = f|_{[0,1]^d} (x)\}.\]

  Since we know $R = \bigsqcup _{k\in\mathbb{Z}}[k, k + 1)$, it follows that 
  \[\Gamma' = \bigsqcup_{k\in\mathbb{Z}}\{(x, y) \in [0, 1]^d \times [k, k + 1) : y = f|_{[0,1]^d} (x)\}.\]
  
  Again, it is sufficient to prove that $|\Gamma''|_e = 0$, where 
  \[\Gamma'' = \{(x, y) \in [0, 1]^d \times [0, 1) : y = f|_{[0,1]^d} (x)\}.\]
  
  For every $n \in \mathbb{N}$, we have $[0, 1) = \bigsqcup_{j=1}^n I_j$, where $I_j = \left[\frac{j - 1}{n}, \frac{j}{n}\right)$ for all $j \in \{1, 2, \ldots, n\}$.
  
  Since we know 
 \[\Gamma'' = \bigsqcup_{j=1}^n \left\{(x, y) \in [0, 1]^d \times I_j : y = f|_{[0,1]^d} (x)\right\},\]
  
  and $f|_{[0,1]^d}$ is measurable on $[0, 1]^d$, it follows that
  \begin{align*}
  |\Gamma''|_e &\le \sum_{j=1}^n \bigg|\left\{(x, y) \in [0, 1]^d \times I_j : y = f|_{[0,1]^d} (x)\right\}\bigg|_e \\
  &\le \sum_{j=1}^n \bigg|f|_{[0,1]^d}^{\text{pre}} (I_j) \times I_j \bigg|_e \\
  &\le \sum_{j=1}^n \bigg|f|_{[0,1]^d}^{\text{pre}} (I_j)\bigg| \cdot |I_j| \\
  &= \frac{1}{n} \sum_{j=1}^n \bigg|f|_{[0,1]^d}^{\text{pre}} (I_j)\bigg| \\
  &= \frac{1}{n} \bigg|\bigsqcup_{j=1}^n f|_{[0,1]^d}^{\text{pre}} (I_j)\bigg| \\
  &= \frac{1}{n} \bigg| f|_{[0,1]^d}^{\text{pre}} \left([0,1)\right)\bigg|  \le \frac{1}{n} \bigg|[0, 1]^d\bigg| = \frac{1}{n}
  \end{align*}

  for all $n \in \mathbb{N}$. Hence, we have

  \begin{equation*}
  |\Gamma''|_e \le \lim_{n \to \infty} \frac{1}{n} = 0.
  \end{equation*}

  Therefore, we obtain
  % \begin{equation*}
  \(|\Gamma''|_e = 0.\)
  % \end{equation*}
    
\end{solution}

\end{document}