\documentclass[UTF8,a4paper,10pt]{article}

\input{preamble.tex}

\begin{document}

% Q1

  \begin{Problem}[]{Zygmund p58 exercise 01}

    (a) There is an analogue for bases different from 10 of usual decimal expansion of number. If $b$ is an integer larger than 1 and $0 < x < 1$,
    show that there exist integral coefficient $c_k$, $0 \leq c_k < b$, such that $x = \sum_{k=1}^{\infty}c_k b^{-k}$. Furthermore, show that expansion is unique unless
    $x = cb^{-k}$
    , in which case there are two expansions.  


    (b) When $b = 3$, the expansion is called the triadic or ternary
    expansion of $x$. Show that Canter set consist of point in $[0,1]$ which has triadic representation such that $c_k$ is either 0 or 2, namely,
    \begin{equation*}
      \begin{aligned}
        \mathcal{C}  = \{x \in [0, 1] : x =\sum_{k=1}^{\infty}c_k 3^{-k}
, c_k \in \{0, 2\}\}
      \end{aligned}
    \end{equation*}
  \end{Problem}

  \begin{mybox}{Random Walks on Graphs}
    We define the following terms\footcite[][133]{motwani95}:
\begin{definition}[Hitting time]
  % Hitting time  \(h_{uv}\)
  The hitting time \(h_{uv}\) (sometimes called the mean first passage time) is the expected number of steps in a random walk that starts at \(u\) and ends upon first reaching \(v\).
\end{definition}
\begin{definition}[Commute time]
  We define \(C_{uv}\), the commute time between \(u\) and \(v\), to be \(C_{uv} = h_{uv}+h_{vu} = C_{vu}\). This is the expected time for a random walk starting at \(u\) to return to \(u\) after at least one visit to \(v\).
\end{definition}
\begin{definition}[Cover time]
  Let \(C_u(G)\) denote the expected length of a walk that starts at \(u\) and ends upon visiting every vertex in \(G\) at least once. The cover time of \(G\), denoted \(C(G)\), is defined by \(C(G) = \max_u C_u(G)\).
\end{definition}
% \dotfill
  \end{mybox}


  \begin{solution}

    \begin{equation*}
      \begin{aligned}
        mR(G) \leq C(G) \leq 32mR(G) \log_2 n + 1.
      \end{aligned}
    \end{equation*}
 

  \end{solution}
  \pagebreak


  \begin{Problem}[]{Zygmund p58 exercise 03}

    Construct a two-dimensional Cantor set in the unit square $\{(x, y) : 0 \leq x, y \leq 1\}$ as follows. Subdivide the square into nine equal parts and keep only the four closed corner squares, removing the remaining region (which forms a cross). Then repeat this process in a suitably scaled version for the remaining squares, ad infinitum. Show that the resulting set is perfect, has plane measure zero, and equals $\mathcal{C} \times \mathcal{C}$.
  
  \end{Problem}


  \begin{solution}\,\\
    Let \(D_0\) be the unit square \(\{(x,y): 0\leq (x,y)\leq 1\}\). Let \(D_k\) be the set remaining after \(i\) steps. Let \(D = \bigcap_{k=1}^{\infty} D_k \) be the resulting set.

    

    (a) To prove that it is a perfect set, we need to show that every point in the set is a limit point of the set.
    \begin{mybox}{Theorem 1.7}
      \begin{enumerate}[label=(\roman*)]
        \item The intersection of any number of closed sets is closed.
        \item The union of any number of open sets is open.
      \end{enumerate}
    \end{mybox}
    Since each \(D_k\) is closed, it follows from Theorem 1.7 that \(D\) is closed. Note that \(D_k\) consists of \(4^k\) closed disjoint intervals, each of which 


    (b)\\
    Since \(D\) is covered by the intervals in any \(D_k\), we have
    \begin{equation*}
      \begin{aligned}
        \left\lvert D\right\rvert_e \leq \left\lvert D_k \right\rvert_e = \left(\frac{4}{9}\right)^{k}.
      \end{aligned}
    \end{equation*}
    Let \(k\to\infty\), we have \(\left\lvert D\right\rvert_e = 0\).


    (c)
    \begin{equation*}
      \begin{aligned}
        D := \bigcap_{k=1}^{\infty} D_k := \bigcap_{k=1}^{\infty} C_k \times C_k =  \left(\bigcap_{k=1}^{\infty} C_k\right) \times \left(\bigcap_{k=1}^{\infty} C_k\right) = C\times C.
      \end{aligned}
    \end{equation*}

    To prove that \(\bigcap_{k=1}^{\infty} C_k \times C_k = \left(\bigcap_{k=1}^{\infty} C_k\right) \times \left(\bigcap_{k=1}^{\infty} C_k\right) \), start with the inclusion from left to right:

For all
    \begin{equation*}
      \begin{aligned}
          (x,y)\in\bigcap_{k=1}^{\infty} C_k \times C_k
      \end{aligned}
    \end{equation*}
    we have
    \begin{equation*}
      \begin{aligned}
        (x,y)\in C_k \times C_k, \forall k \in \mathbb{N}.
      \end{aligned}
    \end{equation*}

    By the definition of Cartesian product, we have
    \begin{equation*}
      \begin{aligned}
        x\in C_k, \forall k \in \mathbb{N},\\
        y\in C_k, \forall k \in \mathbb{N}.
      \end{aligned}
    \end{equation*}
    Thus,
    \begin{equation*}
      \begin{aligned}
        x\in \bigcap_{k=1}^{\infty} C_k,\\
        y\in \bigcap_{k=1}^{\infty} C_k.
      \end{aligned}
    \end{equation*}
    Therefore,
    \begin{equation*}
      \begin{aligned}
        (x,y)\in \left(\bigcap_{k=1}^{\infty} C_k\right) \times \left(\bigcap_{k=1}^{\infty} C_k\right),\forall (x,y)\in D.
      \end{aligned}
    \end{equation*}

    Next, prove the inclusion from right to left. For all
    \begin{equation*}
      \begin{aligned}
          (x,y)\in \left(\bigcap_{k=1}^{\infty} C_k\right) \times \left(\bigcap_{k=1}^{\infty} C_k\right)
      \end{aligned}
    \end{equation*}
    we have
    \begin{equation*}
      \begin{aligned}
        x\in C_k, \forall k \in \mathbb{N},\\
        y\in C_k, \forall k \in \mathbb{N}.
      \end{aligned}
    \end{equation*}
    By the definition of Cartesian product, this implies that
    \begin{equation*}
      \begin{aligned}
        (x,y)\in C_k \times C_k, \forall k \in \mathbb{N}.
      \end{aligned}
    \end{equation*}
    Thus, \((x,y)\in \bigcap_{k=1}^{\infty} C_k \times C_k \)  since it satisfies the definition of intersection.

    Therefore, we can conclude that the two sets are equal:

    \begin{equation*}
      \begin{aligned}
        \bigcap_{k=1}^{\infty} C_k \times C_k =  \left(\bigcap_{k=1}^{\infty} C_k\right) \times \left(\bigcap_{k=1}^{\infty} C_k\right) .
      \end{aligned}
    \end{equation*}




 

  \end{solution}
  
  \pagebreak


  \begin{Problem}[]{Zygmund p59 exercise 04}

    Construct a subset of $[0, 1]$ in the same manner as the Cantor set by
    removing from each remaining interval a subinterval of relative length
    $\theta, 0 < \theta < 1$. Show that the resulting set is perfect and has measure zero.
  
  \end{Problem}

\end{document}