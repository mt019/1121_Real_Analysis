\documentclass[UTF8,a4paper,10pt]{article}

% \documentclass[UTF8,a4paper,14pt]{article}
% \usepackage[utf8]{inputenc}
\usepackage{amsmath}
% \usepackage{algorithm,algorithmic}
\usepackage[linesnumbered,ruled,vlined]{algorithm2e}

% \usepackage{algorithmicx}
% \usepackage{algpseudocode}
\usepackage{hyperref}

% \usepackage{algpseudocode}
\usepackage{amssymb}
\usepackage{amsfonts}
%for 字體
%https://tug.org/FontCatalogue/
% \usepackage[T1]{fontenc}
% \usepackage{tgbonum}
% \usepackage[bitstream-charter]{mathdesign}
% \usepackage[T1]{fontenc}
% \usepackage{bm}#粗體
%\usepackage{boondox-calo}
\usepackage{textcomp}
\usepackage{fancyhdr}%导入fancyhdrf包
\usepackage{ctex}%导入ctex包
\usepackage{enumitem} %for在Latex使用條列式清單
\usepackage{varwidth}
\usepackage{soul} %for \ul
\usepackage{comment}%\begin{comment}\end{comment}
\usepackage{cancel}%\cancel{}
%\usepackage{unicode-math}

\usepackage[dvipsnames, svgnames, x_11names]{xcolor}

\usepackage[low-sup]{subdepth}
\usepackage{subdepth}

\newcommand{\indep}{\Perp \!\!\! \Perp}

\usepackage{amsthm}
\DeclareMathOperator{\E}{\mathbb{E}}
\DeclareMathOperator{\Var}{\textbf{Var}}
\DeclareMathOperator{\Cov}{\textbf{Cov}}
\DeclareMathOperator{\Cor}{\textbf{Cor}}
\DeclareMathOperator{\X}{\mathbf{X}}
\DeclareMathOperator{\Pro}{\mathbf{P}}
\DeclareMathOperator{\M}{\mathbf{M}}
\DeclareMathOperator{\Id}{\mathbf{I}}
\DeclareMathOperator{\Y}{\mathbf{Y}}
\DeclareMathOperator{\MSFE}{\mathbf{MSFE}}
\DeclareMathOperator{\e}{\mathbb{e}}
\DeclareMathOperator{\V}{\mathbf{V}} 
\DeclareMathOperator{\tr}{\text{tr}}
\DeclareMathOperator{\A}{\textbf{A}}
\DeclareMathOperator{\diag}{diag}
\DeclareRobustCommand{\rchi}{{\mathpalette\irchi\relax}}
\newcommand{\irchi}[2]{\raisebox{\depth}{$#1\chi$}} % inner command, used by \rchi
\DeclareMathOperator*{\argmax}{arg\,max}
\DeclareMathOperator*{\argmin}{arg\,min}

\DeclareMathSizes{20}{10}{10}{5}

\usepackage[a4paper, margin=1in]{geometry}
% \setlength\parskip{5ex}% it would be better define distance in ex (5ex) 
                         %  or in pt, pc, mm, etc (see edit below)

\setlength{\parindent}{0pt}
\usepackage{array, makecell} %


%中英文設定
%\usepackage{fontspec}
% \setmainfont{TeX Gyre Termes}
% \usepackage{xeCJK} %引用中文字的指令集
% %\setCJKmainfont{PMingLiU}
% \setCJKmainfont{DFKai-SB}





% \setmainfont{Times New Roman}
% \setCJKmonofont{DFKai-SB}
\pagenumbering{arabic}%设置页码格式
\pagestyle{fancy}
\fancyhead{} % 初始化页眉
\usepackage{advdate}

% \newcommand{\yesterday}{{\AdvanceDate[-1]\today}}

\fancyhead[C]{Real Analysis\quad HW 03\quad  R10A21126\quad  WANG YIFAN\quad   \today}
%\fancyhead[LE]{\textsl{\rightmark}}
%\fancyfoot{} % 初始化页脚
%\fancyfoot[LO]{奇数页左页脚}
%\fancyfoot[LE]{偶数页左页脚}
%\fancyfoot[RO]{奇数页右页脚}
%\fancyfoot[RE]{偶数页右页脚}

% \title{{Econometrics HW 05}}
% \author{R10A21126}
% \date{\today}

%\fancyhf{}
\usepackage{lastpage}
\cfoot{Page \thepage \hspace{1pt} of\, \pageref{LastPage}}

\renewcommand{\headrulewidth}{0.1pt}%分隔线宽度4磅
%\renewcommand{\footrulewidth}{4pt}

\allowdisplaybreaks
\usepackage[english]{babel}
%\usepackage{amsthm}
\newtheorem{theorem}{Theorem}[section]
\newtheorem{corollary}{Corollary}[theorem]
\newtheorem{lemma}[theorem]{Lemma}


\usepackage[most]{tcolorbox}

\definecolor{babyblue}{rgb}{0.54, 0.81, 0.94}

\newtcolorbox[auto counter]{mybox}[1]{
  % Define a new tcolorbox style with custom paragraph spacing
  before upper={\parskip=10pt},
    after upper={\parskip=10pt},
    enhanced,
    arc= 1 mm,boxrule=1.5pt,
    colframe=babyblue!80!pink,
    colback=white,
    coltitle=black,
    % colback=blue!5!white,
    attach boxed title to top left=
    {xshift=1.5em,yshift=-\tcboxedtitleheight/2},
    boxed title style={size=small,
    % frame hidden,
    colback=White},
    top=0.15in,
    % fonttitle=\bfseries,
    title= {#1},
    breakable
  }

\newtcolorbox[auto counter]{Problem}[2][]{
    enhanced,drop shadow={Pink!50!white},
    colframe=pink!80!white,
    fonttitle=\bfseries,
    title=Problem ~\thetcbcounter. #2,
    %separator sign={.},
    coltitle=black,
    colback=pink!15,
    top=0.15in,
    breakable
  }

\newenvironment{solution}
  {\renewcommand\qedsymbol{$\blacksquare$}\begin{proof}[Solution]}
  {\end{proof}}

\theoremstyle{definition}
\newtheorem{definition}{Definition}[section]

%\theoremstyle{notation}
\newtheorem*{notation}{\underline{Notation}}
%\newtheorem*{convention}{\underline{Convention}}
\newtheorem*{convention}{\underline{Convention}}

\theoremstyle{remark}
\newtheorem*{remark}{Remark}

\newenvironment{amatrix}[2]{%% [2] for 2 parameters 
  \left[\begin{array}
    %{cc\,|\,cc}
    %  {@{}*{#2}{c}\,|\,c*{#1}{c}}
     {{}*{#1}{c}\,|\,c*{#2}{c}}
}{%
  \end{array}\right]
}
% For augmented matrix  
%https://tex.stackexchange.com/questions/2233/whats-the-best-way-make-an-augmented-coefficient-matrix


% defines the paragraph spacing
\setlength{\parskip}{0.5em}


\usepackage[sorting=none, citestyle=verbose-inote,backref=true,ibidtracker=context,mincrossrefs=99,backend=biber, 
url = false,
doi = false, isbn=false,]{biblatex}

\addbibresource{R10A21126.bib}

\usepackage{graphicx}
\graphicspath{ {images/} }
\usepackage{caption}

% global change
\SetKwInput{KwData}{Input}
\SetKwInput{KwResult}{Output}
% https://tex.stackexchange.com/questions/299771/how-do-i-rename-data-from-kwdata-and-result-from-kwresult-in-begi

\hypersetup{hidelinks}

\begin{document}

% Q1

  \begin{Problem}[]{Zygmund p58 exercise 01}

    (a) There is an analogue for bases different from 10 of usual decimal expansion of number. If $b$ is an integer larger than 1 and $0 < x < 1$,
    show that there exist integral coefficient $c_k$, $0 \leq c_k < b$, such that $x = \sum_{k=1}^{\infty}c_k b^{-k}$. Furthermore, show that expansion is unique unless
    $x = cb^{-k}$
    , in which case there are two expansions.  


    (b) When $b = 3$, the expansion is called the triadic or ternary
    expansion of $x$. Show that Canter set consist of point in $[0,1]$ which has triadic representation such that $c_k$ is either 0 or 2, namely,
    \begin{equation*}
      \begin{aligned}
        \mathcal{C}  = \{x \in [0, 1] : x =\sum_{k=1}^{\infty}c_k 3^{-k}, c_k \in \{0, 2\}\}.
      \end{aligned}
    \end{equation*}
  \end{Problem}

  \begin{mybox}{Cantor Set}
    \footcite[][42-43]{Wheeden_Zygmund_2015}Consider the closed interval \([0, 1]\). The first stage of the construction is to
subdivide \([0, 1]\) into thirds and remove the interior of the middle third; that
is, remove the open interval \(\left(\frac{1}{3},\frac{2}{3}\right)\). Each successive step of the construction is essentially the same. Thus, at the second stage, we subdivide each of the
remaining two intervals \(\left[0,\frac{1}{3}\right]\)
and  \(\left[\frac{2}{3},1\right]\) into thirds and remove the interiors, \(\left(\frac{1}{9},\frac{2}{9}\right)\) and \(\left(\frac{7}{9},\frac{8}{9}\right)\)
, of their middle thirds. We continue the construction for each of the remaining intervals. The subset of [0, 1] that remains after infinitely many such operations is
called the Cantor set \(C\): thus, if \(C_k\) denotes the union of the intervals left at the
$k$-th stage, then
\begin{equation*}
  \begin{aligned}
    C =  \bigcap_{k=1}^{\infty} C_k  .
  \end{aligned}
\end{equation*}

  \end{mybox}

  \begin{mybox}{Limit Point}

 
    A point \(x\) is a limit point of the set \(E\) if every neighborhood of \(x\)
contains a point \(x\neq y\) such that \(y\in E\).\footcite[][32]{rudin1976principles}  


In other words, \(x\) is a limit point of \(E\) if \(\exists\) a sequence \(\{x_n\}\in E\), s.t. \(x_n\to x\) and \(x_n\neq x\).    \footcite[][3-4]{Wheeden_Zygmund_2015} 
  \end{mybox}

  \begin{mybox}{Perfect Set}


    A closed set \(E\) is said to be a perfect set if every point of \(E\) is a limit point of \(E\).    \footcite[][7]{Wheeden_Zygmund_2015}

    In other words, A closed set \(E\) is said to be a perfect set if \(\forall x\in E, \forall \epsilon>0, (B(x,\epsilon)\setminus \{x\})\cap E\neq \emptyset \).

  \end{mybox}


  \begin{mybox}{Theorem 1.7}
    \begin{enumerate}[label=(\roman*)]
      \item The intersection of any number of closed sets is closed.
      \item The union of any number of open sets is open.
    \end{enumerate}
  \end{mybox}



  \begin{mybox}{Cantor Set is perfect}
    To prove that Cantor Set \(C\) is a perfect set, we need to show that it is closed and every point in the set is a limit point of the set.

    Since each \(C_k\) is closed, it follows from Theorem 1.7 that \(C\) is closed. 

    Then show that every point in \(C\) is a limit point of the set:

    Case 1. Let \(x\in C\) be an endpoint of the interval \(I_k\subseteq C_k\). Consider the intervals \(I_k^i\ \subseteq C_{k+i}\) with endpoint \(x\), let \(x_1\) be the other endpoint of \(I_k^1\subseteq C_{k+1}\), \(x_2\) be the other endpoint of \(I_k^2\subseteq C_{k+2}\),..., \(x_n\) be the other endpoint of \(I_k^n \subseteq C_{k+n}\). 
    Thus, \(|x_n-x|=(\frac{1}{3})^{k+n}\).

    We have 
    \begin{equation*}
      \begin{aligned}
        x_n\to x,\\
        x_n \neq x,\\
        x_n \in C.
      \end{aligned}
    \end{equation*}
    Therefore, \(x\) is a limit point of \(C\).

    Case 2. Suppose \(x\in C\) is not an endpoint of any interval consisting \(C\). \(\forall n\in\mathbb{N}\), we have \(x\in (a_n, b_n)\), where \(a_n\in C\) and \(|a_n-x|<(\frac{1}{3})^{n}\). Let \(x_n = a_n\), \(\{x_n\}\) is the squence s.t. 
    \begin{equation*}
      \begin{aligned}
        x_n\to x,\\
        x_n \neq x,\\
        x_n \in C.
      \end{aligned}
    \end{equation*}
    Thus, \(x\) is a limit point of \(C\).

    We can conclude that every point in \(C\) is a limit point of the set. Therefore, \(C\) is perfect.

    
  \end{mybox}


  \begin{solution}

    \begin{equation*}
      \begin{aligned}
      \end{aligned}
    \end{equation*}
 

  \end{solution}
  \pagebreak


  \begin{Problem}[]{Zygmund p58 exercise 03}

    Construct a two-dimensional Cantor set in the unit square $\{(x, y) : 0 \leq x, y \leq 1\}$ as follows. Subdivide the square into nine equal parts and keep only the four closed corner squares, removing the remaining region (which forms a cross). Then repeat this process in a suitably scaled version for the remaining squares, ad infinitum. Show that the resulting set is perfect, has plane measure zero, and equals $\mathcal{C} \times \mathcal{C}$.
  
  \end{Problem}


  \begin{solution}\,\\
    Let \(D_0\) be the unit square \(\{(x,y): 0\leq (x,y)\leq 1\}\). Let \(D_k\) be the set remaining after \(i\) steps. Let \(D = \bigcap_{k=1}^{\infty} D_k \) be the resulting set.

  

    (a) Show that \(D\) has plane measure
    zero:

    Since \(D\) is covered by the intervals in any \(D_k\), we have
    \begin{equation*}
      \begin{aligned}
        \left\lvert D\right\rvert_e \leq \left\lvert D_k \right\rvert_e = \left(\frac{4}{9}\right)^{k}.
      \end{aligned}
    \end{equation*}
    Let \(k\to\infty\), we have \(\left\lvert D\right\rvert_e = 0\).


    (b)
    \begin{equation*}
      \begin{aligned}
        D &:= \bigcap_{k=1}^{\infty} D_k &\text{(By definition}\\
        &= \bigcap_{k=1}^{\infty} C_k \times C_k &\text{(By definition}\\
        &=  \left(\bigcap_{k=1}^{\infty} C_k\right) \times \left(\bigcap_{k=1}^{\infty} C_k\right) &\text{(To be proved}\\
        &= C\times C&\text{(By definition}.
      \end{aligned}
    \end{equation*}

    To prove that \(\bigcap_{k=1}^{\infty} C_k \times C_k = \left(\bigcap_{k=1}^{\infty} C_k\right) \times \left(\bigcap_{k=1}^{\infty} C_k\right) \), start with the inclusion from left to right:

For all
    \begin{equation*}
      \begin{aligned}
          (x,y)\in\bigcap_{k=1}^{\infty} C_k \times C_k
      \end{aligned}
    \end{equation*}
    we have
    \begin{equation*}
      \begin{aligned}
        (x,y)\in C_k \times C_k, \forall k \in \mathbb{N}.
      \end{aligned}
    \end{equation*}

    By the definition of Cartesian product, we have
    \begin{equation*}
      \begin{aligned}
        x\in C_k, \forall k \in \mathbb{N},\\
        y\in C_k, \forall k \in \mathbb{N}.
      \end{aligned}
    \end{equation*}
    Thus,
    \begin{equation*}
      \begin{aligned}
        x\in \bigcap_{k=1}^{\infty} C_k,\\
        y\in \bigcap_{k=1}^{\infty} C_k.
      \end{aligned}
    \end{equation*}
    Therefore,
    \begin{equation*}
      \begin{aligned}
        (x,y)\in \left(\bigcap_{k=1}^{\infty} C_k\right) \times \left(\bigcap_{k=1}^{\infty} C_k\right),\forall (x,y)\in D,
      \end{aligned}
    \end{equation*}
    which means
    \begin{equation*}
      \begin{aligned}
        \bigcap_{k=1}^{\infty} C_k \times C_k \subseteq  \left(\bigcap_{k=1}^{\infty} C_k\right) \times \left(\bigcap_{k=1}^{\infty} C_k\right).
      \end{aligned}
    \end{equation*}


    Next, prove the inclusion from right to left. For all
    \begin{equation*}
      \begin{aligned}
          (x,y)\in \left(\bigcap_{k=1}^{\infty} C_k\right) \times \left(\bigcap_{k=1}^{\infty} C_k\right)
      \end{aligned}
    \end{equation*}
    we have
    \begin{equation*}
      \begin{aligned}
        x\in C_k, \forall k \in \mathbb{N},\\
        y\in C_k, \forall k \in \mathbb{N}.
      \end{aligned}
    \end{equation*}
    By the definition of Cartesian product, this implies that
    \begin{equation*}
      \begin{aligned}
        (x,y)\in C_k \times C_k, \forall k \in \mathbb{N}.
      \end{aligned}
    \end{equation*}
    Since it satisfies the definition of intersection,we have \((x,y)\in \bigcap_{k=1}^{\infty} C_k \times C_k \), implying that 
    \begin{equation*}
      \begin{aligned}
        \left(\bigcap_{k=1}^{\infty} C_k\right) \times \left(\bigcap_{k=1}^{\infty} C_k\right) \subseteq \bigcap_{k=1}^{\infty} C_k \times C_k .
      \end{aligned}
    \end{equation*}
    


    Therefore, we can conclude that the two sets are equal:

    \begin{equation*}
      \begin{aligned}
        \bigcap_{k=1}^{\infty} C_k \times C_k =  \left(\bigcap_{k=1}^{\infty} C_k\right) \times \left(\bigcap_{k=1}^{\infty} C_k\right) .
      \end{aligned}
    \end{equation*}

    (c) Prove that it is a perfect set:
    
    Since \(C\) is perfect, \(D=C\times C\) is perfect, by the property of perfect set.


 

  \end{solution}
  
  \pagebreak


  \begin{Problem}[]{Zygmund p59 exercise 04}

    Construct a subset of $[0, 1]$ in the same manner as the Cantor set by
    removing from each remaining interval a subinterval of relative length
    $\theta, 0 < \theta < 1$. Show that the resulting set is perfect and has measure zero.
  
  \end{Problem}

  \begin{solution}\,

  If \(C'_k\) denotes the union of the intervals left at the
$k$-th stage, then the resulting set is
\begin{equation*}
  \begin{aligned}
    C' =  \bigcap_{k=1}^{\infty} C'_k  .
  \end{aligned}
\end{equation*}


  (a) To prove that the set \(C'\) is a perfect set, we need to show that it is closed and every point in the set is a limit point of the set.

  Since each \(C'_k\) is closed, it follows from Theorem 1.7 that \(C'\) is closed. 

  Then show that every point in \(C'\) is a limit point of the set:

  Case 1. Let \(x\in C'\) be an endpoint of \(I_k\subseteq C'_k\). Consider the intervals \(I_k^i\ \subseteq C'_{k+i}\) with endpoint \(x\), let \(x_1\) be the other endpoint of \(I_k^1\subseteq C'_{k+1}\), \(x_2\) be the other endpoint of \(I_k^2\subseteq C'_{k+2}\),..., \(x_n\) be the other endpoint of \(I_k^n
  \subseteq C'_{k+n}\).
  Thus, \(|x_n-x|=(\frac{1}{3})^{k+n}\).

  We have the squence\(\{x_n\}\) s.t. 
  \begin{equation*}
    \begin{aligned}
      x_n&\to x,\\
      x_n&\neq x,\\
      x_n&\in C'.
    \end{aligned}
  \end{equation*}
  Therefore, \(x\) is a limit point of \(C'\).

  Case 2. Suppose \(x\in C'\) is not an endpoint of any interval consisting \(C'\). \(\forall n\in\mathbb{N}\), we have \(x\in (a_n, b_n)\), where \(a_n\in C'\) and \(|a_n-x|<(\frac{1-\theta}{2})^{n}\). Let \(x_n = a_n\), then \(\{x_n\}\) is the squence s.t. 
  \begin{equation*}
    \begin{aligned}
      x_n&\to x,\\
      x_n&\neq x,\\
      x_n&\in C'.
    \end{aligned}
  \end{equation*}
  Thus, \(x\) is a limit point of \(C'\).

  We can conclude that every point in \(C'\) is a limit point of the set. Therefore, \(C'\) is perfect.

  \dotfill

  (b) Show that \(C'\) has plane measure
  zero:

  Since \(C'\) is covered by the intervals in any \(C'_k\), we have
  \begin{equation*}
    \begin{aligned}
      \left\lvert C'\right\rvert_e \leq \left\lvert C'_k \right\rvert_e = \left(1-\theta\right)^{k}.
    \end{aligned}
  \end{equation*}
  Let \(k\to\infty\), we have \(\left\lvert C'\right\rvert_e = 0\).

\end{solution}

\end{document}